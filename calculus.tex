\subsection{Closed Form Solutions}
Theorem \ref{thm:main} implies that optimal contracts have simple functional forms in parametric settings. Here we describe the solutions for common parametrization and note their properties.

Optimal contracts depend on effort cost, risk preferences (which determine the link function $g$), and on the distribution of output (which determines the score $S$). Tables \ref{tab:utility_functions} and \ref{tab:dists} provide basic formulae for the link and score functions that make up optimal contracts $g(\mu + \lambda S(y | a_0))$.

\begin{table}[ht]
    \centering
    \caption{Utility Functions, Link Functions, and Wage Functions}
    \label{tab:utility_functions}
    \renewcommand{\arraystretch}{1.5} % Adjust row height for better spacing
    \setlength{\tabcolsep}{10pt} % Slightly reduced column spacing
    \begin{tabular*}{\textwidth}{@{\extracolsep{\fill}}lccc}
        \toprule
        & \multicolumn{1}{c}{Utility Function} & \multicolumn{1}{c}{Link Function} & \multicolumn{1}{c}{Wage Function} \\ 
        & \( u(x) \) & \( g(z) \) & \( w(z) \) \\ 
        \midrule
        Log  & \multicolumn{1}{l}{\( \log(x + w_0) \)} & \multicolumn{1}{l}{\( \log(\max(w_0, z)) \)} & \multicolumn{1}{l}{\( (z - w_0)^+ \)} \\ 
        CRRA & \multicolumn{1}{l}{\( \frac{(x + w_0)^{1-\gamma}}{1-\gamma} \)} & \multicolumn{1}{l}{\( \frac{\max(w_0^\gamma, z)^{\frac{1-\gamma}{\gamma}}}{1-\gamma} \)} & \multicolumn{1}{l}{\( \left( (z^+)^{\frac{1}{\gamma}} - w_0 \right)^+ \)} \\ 
        CARA & \multicolumn{1}{l}{\( \frac{-\exp(-\alpha (x + w_0))}{\alpha} \)} & \multicolumn{1}{l}{\( -\frac{1}{\alpha \max(\exp(\alpha w_0), z)} \)} & \multicolumn{1}{l}{\( \frac{(\log^+ z - \alpha w_0)^+}{\alpha} \)} \\ 
        \bottomrule
    \end{tabular*}
    \captionsetup{font=footnotesize} % Match figure's note formatting
    \caption*{\textit{Note:} The utility function is the agent's utility from consumption given starting wealth $w_0$ and a transfer, $x$. The link and wage functions are in terms of $z$, which is a function of the outcome, $y$: \( z(y) = \lambda + \mu S(y|a_0) \).}
\end{table}
 

\begin{table}[ht]
    \centering
    \caption{Error Distributions}
    \label{tab:dists}
    \renewcommand{\arraystretch}{1.5} % Reduce vertical spacing
    \small % Reduce font size further
    \setlength{\tabcolsep}{6pt} % Reduce column spacing for compactness
    \resizebox{\textwidth}{!}{ % Ensures table fits within page width
    \begin{tabular}{l p{5.5cm} p{3.8cm} p{3.8cm}} % Keep all columns aligned and tight
        \toprule
        \textbf{Distribution} & \textbf{Probability Density} & \textbf{Score Function \( S(y|a) \)} & \textbf{Mean} \\ 
        \midrule
        Gaussian
        & \( \mathcal{N}(a, \sigma^2) \)
        & \( \frac{y - a}{\sigma^2} \) 
        & \( a \) 
        \\ 

        Log Normal
        & \( \frac{1}{y \sqrt{2\pi\sigma^2}} \exp \!\Bigl( -\frac{(\log(y)-a)^2}{2\sigma^2} \Bigr) \) 
        & \( \frac{\log(y) - a}{\sigma^2} \) 
        & \( \exp\!\Bigl(a + \frac{\sigma^2}{2}\Bigr) \) 
        \\ 

        Poisson
        & \( \frac{a^y e^{-a}}{y!} \) 
        & \( \frac{y - a}{a} \) 
        & \( a \) 
        \\ 

        Exponential 
        & \( \frac{1}{a} e^{-\tfrac{y}{a}} \) 
        & \( \frac{y-a}{a^2} \) 
        & \( a \) 
        \\ 

        Bernoulli
        & \( a^y (1 - a)^{1 - y} \), \( y \in \{0, 1\} \) 
        & \( \frac{y-a}{a-a^2} \)
        & \( a \) 
        \\  

        Geometric
        & \( \Bigl(1 - \frac{1}{a}\Bigr)^{y - 1} \Bigl(\frac{1}{a}\Bigr) \), \( y \in \{1,2,\dots\} \) 
        & \( \frac{y - a}{a^2 - a} \) 
        & \( a \) 
        \\ 

        Binomial
        & \( \binom{n}{y} a^y (1 - a)^{n - y} \), \( y \in \{0,\dots,n\} \) 
        & \( \frac{y - na}{a - a^2} \) 
        & \( n a \) 
        \\ 

        Gamma
        & \( f(y \mid n, a) = \frac{y^{n - 1} e^{-\tfrac{y}{a}}}{\Gamma(n)\, a^n} \) 
        & \( \frac{y - n a}{a^2} \) 
        & \( n a \)
        \\ 

        Student's \(t\)
        & \( 
        \frac{\Gamma\!\bigl(\tfrac{\nu + 1}{2}\bigr)}{\Gamma\!\bigl(\tfrac{\nu}{2}\bigr)\,\sqrt{\pi\nu}\,\sigma}
        \left(1 + \frac{1}{\nu} \,\frac{(y - a)^2}{\sigma^2}\right)^{-\tfrac{\nu + 1}{2}} 
        \) 
        & \( \frac{(\nu+1)(y-a)}{\nu\,\sigma^2 + (y-a)^2} \) 
        & \( a \) 
        \\ 

        Exponential Family
        & \( h(y)\,\exp\Bigl(\eta(a)\,T(y) - A(a)\Bigr) \) 
        & \( T(y)\,\frac{d\eta(a)}{da} \;-\; \frac{dA(a)}{da} \) 
        & \textit{(Not specified)} 
        \\ 

        \( y = a + X, X \sim h \)
        & \( g(y - a) \) 
        & \( -\frac{g'(y - a)}{h(y - a)} \) 
        & \( a + \mathbb{E}[X] \) 
        \\ 

        \( y = a X, X \sim h \) 
        & \( \bigl|\tfrac{1}{a}\bigr|\;h\!\Bigl(\tfrac{y}{a}\Bigr) \) 
        & \( -\frac{1}{a} - \frac{y}{a^2} \frac{g'(\frac{y}{a})}{g(\frac{y}{a})} \) 
        & \( a \,\mathbb{E}[X] \)
        \\ 
        \bottomrule
    \end{tabular}
    } % End resizebox
    \captionsetup{font=footnotesize} % Match figure's note formatting
    \caption*{\textit{Note:} This table presents probability the PDF, score function, and means of probability distributions as functions of the agent's chosen action, $a$.}
\end{table}



\textbf{Linearity}
Linear contracts play a prominent role in contract theory. It has long been noted that linear contracts are common, but that they only arise under relatively special assumptions \citep{holmstrom1987aggregation,carroll2015robustness}.

Our model yields piecewise linear contracts in a number of examples, as seen in the tables and in the Gaussian-log utility example. The key ingredients are log utility, which makes the wage function linear in the score, and a linear score function. This includes many distributions, because the score is linear for distributions in an exponential family with linear sufficient statistics. That is, when $f(y, a)$ is of the form
\begin{equation}
\label{eq:linear-exponential-family}
f(y | a)
=
\exp\left(\eta (a) y + A(a)\right) \\
\text{.}    
\end{equation}

We note this as follows:

\begin{remark}[Linear Contracts]
    Assume that utility is log (as in $u(x) = \log(w_0 + x)$) and that the distribution of outcomes is in an exponential family with linear sufficient statistic (as in equation \ref{eq:linear-exponential-family}). Then the relaxed optimal contract wage function is piecewise linear. This includes the Gaussian, exponential, Poisson, and gamma distributions.\footnote{Note that the remark covers distributions with support different than $\mathbb R$, such as the exponential. These distributions do not satisfy Assumption (\ref{assump:regularity_S}). Nevertheless, the proof of proposition \ref{prop:relaxed-optimal-contract} does not use that the support is $\mathbb R$, so the remark holds. However, our proof that the relaxed optimal contract is optimal does use that the support is $\mathbb R$. Therefore, we do not know whether Theorem 1 can be extended to different supports, and thus whether the remark can be extended to optimal contracts in the case of sufficiently high reservation utility. We conjecture based on numerical results that this is true and hope to include these facts in our next revision.}
\end{remark}

\textbf{Concave and Convex Contracts} The optimal contract is always at least partially convex because the limited liability constraint requires that $w(y) = 0$ for all $y$ less than some threshold, $\ubar y$. The region where the limited liability constraint does not bind, however, can be convex or concave depending on the agent's risk aversion.

\begin{remark}
    \label{rem:convexandconcave}
    Suppose the score function is linear, and let $\mu S(y | a_0) = k y$. Then, if the agent has CRRA utility, the region of the optimal contract where the limited liability constraint does not bind is convex if the risk aversion parameter $\gamma > 1$ and concave if $\gamma < 1$:\footnote{Assumption \ref{assump:concavity_inverse_marginal_utility} requires that $\gamma > \frac{1}{2}$.}  
    $$ w(y) = \left( \lambda + k y \right)^{\frac{1}{\gamma}} \text{ for } y > \ubar y. $$  

    If the agent has CARA utility, the same region of the optimal contract is logarithmic:
    $$ w(y) =  \frac{\log( \lambda + k y )}{\alpha} - w_0 \text{ for } y > \ubar y .$$
\end{remark}

The remark shows that the shape of the optimal contract depends heavily on the agent's utility function, and a wide variety of contracts are potentially compatible with the theory. Our result is contrary to \cite{conlon2009two}'s view that convex contracts, like CEO compensation with stock options, are incompatible with the first-order approach. In fact, the optimal contract with CRRA utility and $\gamma < 1$ closely approximates a CEO compensation package of stock options with many strike prices. To see this, consider a CEO who receives $n_{i}$ options with strike price $s_i$ that expire at the end of the year. The CEO's wage is 
$$ w(y) = \sum_i n_{i} (y - strike_i)^+, $$
where $y$ is the stock's end of year price price. 
The wage function's slope at $y$ is $\sum_{strike_i < y} n_{i}$; the wage is a piecewise-linear function where the slope increases as $y$ increases. The options package wage may be a discretization of the contract the model predicts for an agent with CRRA utility with $\gamma < 1$, and a linear score function. Its slope is also increasing on the outcome and equals $\frac{k}{\gamma}(\lambda + k y) ^\frac{1-\gamma}{\gamma}$ for some constant $k$. 

Remark \ref{rem:convexandconcave} affirms \citeauthor{holmstrom1987aggregation}'s (\citeyear{holmstrom1987aggregation}) view that the principle agent model does not robustly predict linear wages. The optimal wage is only linear if the agent's risk aversion parameter is precisely $\gamma=1$. However, if the agent's risk aversion is approximately one, the optimal contract will be approximately linear, and using a linear contract may be practical. 

\textbf{Limitations and numerical methods.}
The formulae have two main limitations. First, there is no general closed form solution for the Lagrange multipliers $\lambda$ and $\mu$. Thus, Theorem \ref{thm:main} guarantees that optimal contracts have the described functional form, but the Lagrange multipliers have to be calculated numerically. Second, as noted in Theorem \ref{thm:main}, these simple formulae do not hold for sufficiently low reservation utility.

To address these limitations, we provide accompanying code to numerically solve for optimal contracts. The numerical methods solve both the relaxed problem, and also the full cost minimization problem, in the case where only a finite number of global constraints bind. This is the case in all the experiments that we conducted. Both are convex optimization problems, and can be solved at trivial computational cost in all our experiments. The code implements all the common specifications in tables \ref{tab:utility_functions} and \ref{tab:dists}.
