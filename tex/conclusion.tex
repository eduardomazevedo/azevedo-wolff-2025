This paper shows that when an agent's reservation utility is sufficiently high, the longstanding concerns about the validity of the first-order approach largely disappear. Despite the possibility of multiple local maxima under low reservation utility, forcing the principal to provide higher overall utility eliminates global deviations, ensuring both existence of an optimal contract and applicability of the standard calculus-based approach. Our results hold under natural assumptions and extend to many commonly used distributions and preferences. This suggests that the first-order approach is especially relevant in competitive labor markets or other environments where agents have strong outside options.

In particular, our results show that the many existing results in the literature that have been proven assuming the first-order approach hold in this broader array of settings, include many settings with option-like contracts, and even contracts with wages that are convex functions of output. We hope that this result and the basic calculus formulae we provide will be useful in applied and theoretical work.