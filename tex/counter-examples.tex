\subsection{Discussion of Assumptions and Counter-Examples}
\textbf{Unbounded utility.}

Assumption \ref{assump:utility_cost} requires that $u(\infty) = \infty$. Without this assumption, utility may be bounded, as with CARA utility. So, as $\bar U$ converges to infinity, neither the relaxed cost minimization problem nor the cost minimization problem have a solution. In that case, Theorem \ref{thm:main} holds as stated, but is somewhat vacuous, as neither problem has a solution.

Nevertheless, we can say something more interesting about the bounded utility case. Proposition \ref{prop:relaxed-optimal-contract} changes because we have to take feasibility into account. For some actions $a_0$, if $c'(a_0)$ is large enough, it may be impossible to even satisfy the local incentive compatibility constraint. In that case, the Pareto frontier from Figure \ref{fig:pf} is empty. For some actions $a_0$, it is possible to satisfy the local incentive compatibility constraint, but the relaxed cost minimization problem is only feasible for reservation utility $\bar U$ below some threshold $\bar U _ R$. The March 2025 version of this paper shows that there still exists $\bar U^* < \bar U _R$ such that the first-order approach is valid for any $\bar U \geq \bar U^*$. Thus, the results hold in spirit, although the statement of proposition \ref{prop:relaxed-optimal-contract} is slighly more complex to deal with feasibility.

\textbf{Support of the distribution of output.}
Assumption \ref{assump:regularity_S} requires that the support of the distribution of output is the real line. This rules out interesting cases like the exponential distribution, which has a lower bound of zero. This assumption is important in the proof of Lemma \ref{lem:threshold-outcome}, that shows that the kink of the distribution approaches $-\infty$ as $\lambda$ approaches infinity. If the support of $f$ has a finite lower bound, this fails. Instead, it is possible to prove that the kink approaches the finite lower bound. The proof works in the same way, as if this did not happen it would contradict the local incentive compatibility constraint. For this reason, we conjecture that Theorem \ref{thm:main} is false in this case, although we have not found a counterexample.

Regardless, we have found that the first-order approach is still often valid in examples where support has a finite lower bound. For example, in Figure \ref{fig:poisson-log-pp} with the Poisson distribution, the first-order approach is valid for any positive reservation wage. Figure \ref{fig:exponential-log-pp} displays an example with the exponential distribution, where the first-order approach is valid for any reservation wage. Likewise, we have found that discrete distributions like the Poisson also follow the same pattern, even though they do not satisfy the full support assumption.

% Exponential - log utility figure
\begin{figure}[p]
    \centering
    \includegraphics[width=\textwidth]{figures/log-exponential/pp_stacked.pdf}
    \captionsetup{font=footnotesize} % Makes the note footnote-sized
    \caption{Optimal contracts with exponential distribution and log utility.}
    \label{fig:exponential-log-pp}
    \caption*{\textit{Note:} Top panel: optimal wage function $w(y)$. Bottom panel: agent's expected utility $U(v^*, a)$ given optimal contract and action $a$. Colors represent reservation utility. Dashed lines indicate that the first-order approach is invalid at that reservation utility. Dots indicate the recommended action. The thin horizontal line indicates indifference between local maxima. Output has exponential distribution with mean $a$, initial wealth is $50$ (both in thousands of dollars), and the cost function is $c(a) = a^2 / 30000$.}
\end{figure}


\textbf{Increasing score.}

OLD TEXT:
The monotone likelihood ratio property is a critical assumption. Namely, we assume that the score is strictly increasing in $y$. Examples show that this is critical for Theorem \ref{thm:main}. An illustrative example is the fat-tailed Student-$t$ distribution. As shown in table \ref{tab:dists}, the score is not monotone in $y$. Close to the mean, the score is increasing, as higher $y$ is statistical evidence of higher effort. However, once we go out into the tails, an outlier positive or negative $y$ is interpreted as a large shock, and the score is close to zero. Computational examples show that there are cases where the first-order approach is not valid, regardless of the reservation utility.

We make two other assumptions that are quite restrictive, but much less critical. We assume that utility is unbounded ($u(\infty) = \infty$), and that the support of the distribution of output is the real line. Readers may have noted that many of our computational examples violate these two assumptions, but that the first-order approach holds regardless. Indeed, these assumptions are used in very specific places in the proofs, so that the argument works as long as examples are close enough to the assumptions.

The assumption of unbounded $u$ is used in two places. The first place is that it is needed to guarantee that there exists a solution to the cost minimization problem. Otherwise, there are cases like $\bar U > u(\infty) - c(0)$, where no contract satisfies the constraints of the relaxed cost minimization problem. This complicates the statements of the theorems (because we have to deal with the trivial infeasible case), while not adding any substantive content to the results.

The second place is showing that the probability that the agent is paid zero converges to zero as we increase reservation utility (lemma \ref{lem:threshold-outcome}). The fact that the probability converges to zero, and not just some very low positive probability, uses $u(\infty) = \infty$. Thus, one can likely fabricate counterexamples to Theorem \ref{thm:main} by violating this assumption. In practice, however, utility functions like CARA have enough upside to render this limiting probability negligible, so that the first-order approach is often valid.

The assumption of unbounded support is used in only one place. We need it to show that the largest outcome where the agent is paid zero converges to negative infinity. This is referred to as $\ubar y$ in the appendix. $\ubar y$ converging to negative infinity is crucial in the proof of concavity in lemma \ref{lem:inf_lambda_second_deriv}. Thus, one can likely fabricate counterexamples using distributions with a bounded support. In practice, however, real examples like the exponential distributions have lower bounds that are far away, so that the first-order approach holds anyway, following the same logic as in the proof of lemma \ref{lem:inf_lambda_second_deriv}, but with $\ubar y$ far enough towards $-\infty$ even if it does not converge to $-\infty$.