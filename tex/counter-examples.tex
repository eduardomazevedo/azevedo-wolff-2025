\subsection{Discussion of Assumptions and Counter-Examples}
\medskip\noindent\textit{\textbf{Increasing score.}}
Increasing score (also known as monotone likelihood ratio property) is a critical assumption, without which the first-order approach often fails. The importance of the assumption is illustrated in the fat-tailed Student-$t$ distribution. The $t$ distribution is similar to a Gaussian, but has fatter tails. As a consequence, the score is similar to the Gaussian score close to the mean, but approaches zero in both tails (table \ref{tab:dists}). Figure \ref{fig:t-log-pp} displays an example where the first-order approach is invalid for any reservation wage where principal profits are positive. 

% Student-t - log utility figure
\begin{figure}[p]
    \centering
    \includegraphics[width=\textwidth]{figures/log-t-sigma=20.0/pp_stacked.pdf}
    \captionsetup{font=footnotesize} % Makes the note footnote-sized
    \caption{Optimal contracts with Student-$t$ distribution and log utility.}
    \label{fig:t-log-pp}
    \caption*{\textit{Note:} Top panel: optimal wage function $w(y)$. Bottom panel: agent's expected utility $U(v^*, a)$ given optimal contract and action $a$. Colors represent reservation utility. Dashed lines indicate that the first-order approach is invalid at that reservation utility. Dots indicate the recommended action. The thin horizontal line indicates indifference between local maxima. Output has Student-$t$ distribution with mean $a$, scale parameter $\sigma = 20.0$, tail parameter $\nu = 1.15$, initial wealth is $50$ (both in thousands of dollars), and the cost function is $c(a) = a^2 / 30000$.}
\end{figure}


\medskip\noindent\textit{\textbf{Limited liability.}}
Limited liability is crucial to guarantee existence of an optimal contract. The intuition is clear from the Lagrangian's first order condition in equation (\ref{eq:foc}). When the right-hand side is negative, it is optimal to give the agent the lowest possible $v(y)$. This only has a solution with both limited liability and $u(0) > - \infty$. This is why with log utility $u(x) = \log(w_0 + x)$ our assumptions require $w_0 > 0$. Under our assumption that the score is unbounded below, a negative right-hand side happens for any $\mu >0$. These existence issues are well-known since \textcite{mirrlees1999}. Limited liability is not important for guaranteeing the validity of the first-order approach, and in fact the kink in the optimal contract makes it more difficult to guarantee the validity of the first-order approach (see proof of proposition \ref{prop:concave}).

\medskip\noindent\textit{\textbf{Unbounded utility.}}
The assumption of $u$ unbounded above is used in two places. The first is to guarantee that there exists a solution to the cost minimization problem. Otherwise, there are cases where $\bar U > u(\infty) - c(0)$, so that no contract satisfies the constraints of the relaxed cost minimization problem. This complicates the statements of proposition \ref{prop:relaxed-optimal-contract} (because we have to deal with the trivial infeasible case), while not adding much to the results. For example, Theorem \ref{thm:main} is true as stated in this case, but is somewhat vacuous, as neither problem has a solution.

The second place is showing that the probability that the agent is paid zero converges to zero as we increase reservation utility (lemma \ref{lem:threshold-outcome}). The fact that the probability converges to zero, and not just some very low positive probability, uses $u(\infty) = \infty$. In practice, however, utility functions like CARA have enough upside so that often the first-order approach is valid for many reservation wages where the problem is feasible (e.g., Figure \ref{fig:gaussian-cara-pp}).

\medskip\noindent\textit{\textbf{Support of the distribution of output.}}
Assumption \ref{assump:regularity_S} requires that the support of the distribution of output is the real line. This rules out interesting cases like the exponential distribution, which has a lower bound of zero. This assumption is important in the proof of Lemma \ref{lem:threshold-outcome}, that shows that the kink of the distribution approaches $-\infty$ as $\lambda$ approaches infinity. If the support of $f$ has a finite lower bound, this fails. Instead, it is possible to prove that the kink approaches the finite lower bound. The proof works in the same way, as if this did not happen it would contradict the local incentive compatibility constraint. For this reason, it is an open question whether Theorem \ref{thm:main} is false in the case where the support of the distribution of output has a finite lower bound.

Nevertheless, we have found that the first-order approach is often valid in examples where support has a finite lower bound. For example, in Figure \ref{fig:poisson-log-pp} with the Poisson distribution, the first-order approach is valid for nearly any positive reservation wage. Figure \ref{fig:exponential-log-pp} displays an example with the exponential distribution, where the first-order approach is valid for any reservation wage. Likewise, we have found that discrete distributions like the Poisson also follow the same pattern, even though they do not satisfy the full support assumption.

% Exponential - log utility figure
\begin{figure}[p]
    \centering
    \includegraphics[width=\textwidth]{figures/log-exponential/pp_stacked.pdf}
    \captionsetup{font=footnotesize} % Makes the note footnote-sized
    \caption{Optimal contracts with exponential distribution and log utility.}
    \label{fig:exponential-log-pp}
    \caption*{\textit{Note:} Top panel: optimal wage function $w(y)$. Bottom panel: agent's expected utility $U(v^*, a)$ given optimal contract and action $a$. Colors represent reservation utility. Dashed lines indicate that the first-order approach is invalid at that reservation utility. Dots indicate the recommended action. The thin horizontal line indicates indifference between local maxima. Output has exponential distribution with mean $a$, initial wealth is $50$ (both in thousands of dollars), and the cost function is $c(a) = a^2 / 30000$.}
\end{figure}
