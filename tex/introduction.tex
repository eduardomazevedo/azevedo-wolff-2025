One of the workhorse models in economics is the principal-agent problem with moral hazard. The principal hires the agent to take an action $a$ in $\mathbb R _ +$ that affects the distribution $f(y|a)$ of output $y$. The principal can only condition payments on realized output, and designs a contract $w(y) \geq 0$ to provide incentives to the agent. The agent chooses the action $a$ to maximize her utility from wages minus her cost of effort $c(a)$ and the principal chooses the contract to maximize expected profits.

The main solution method for this problem is the first-order approach, which assumes that only local deviations in $a$ are binding, and yields a simple formula for the optimal contract.%
\footnote{%
See \textcite{holmstrom1978incentives}, the excellent survey \textcite{georgiadis2022contracting}, and recent work in \textcite{conlon2009two,kadan2017existence,chade2020no,chaigneau2022should,castro2024disentangling}.
}
Because of tractability, a vast literature simply assumes that the first-order approach is valid, and many applied papers make restrictive assumptions to avoid the issue of non-local deviations.%
\footnote{%
\textcite{mirrlees1999} showed that the first-order approach is not always valid. Examples of papers assuming the first-order approach include \textcite{jewitt2008moral, moroni2014existence, chaigneau2022should,castro2024disentangling}. Virtually all early work, including the seminal papers by \textcite{holmstrom1978incentives, holmstrom1979moral}, and \textcite{zeckhauser1970medical} assumes it. \textcite{holmstrom1979moral} notes that 
``one has to assume that the agent's optimal choice of action is unique
for the optimal [...] This assumption seems very difficult to validate [...] and regrettably we have to leave the question about its validity open.'' In applied work, restrictive assumptions such as linear contracts or binary effort are often made to avoid this issue. \textcite{edmans2009multiplicative} is an example of an important recent paper using both binary effort and linear contracts.
}

Unfortunately, existing sufficient conditions for the first-order approach to be valid are restrictive. The seminal papers are \textcite{rogerson1985} and \textcite{jewitt1988justifying}, followed by an extensive literature.%
\footnote{Important generalizations include \textcite{sinclair1994first,conlon2009two,jung2015information,chade2020no}, and \textcite{jung2024proxy}. \textcite{chaigneau2022should, chaigneau2024theory} develop sufficient conditions with limited liability.}
\textcite{kadan2017existence} summarize the general view that ``conditions facilitating the first-order approach are typically quite demanding.'' The key issues are elegantly explained by \textcite{chaigneau2022should} and \textcite{conlon2009two}, who says ``Unfortunately, the Jewitt conditions are tied to the concavity, not only of the technology, but also of the payment schedule. Thus, there are interesting cases where the Jewitt conditions should fail, since the payment schedule is often not concave. For example, managers often receive stock options, face liquidity constraints.''% 
\footnote{\textcite{conlon2009two} continues: ``
I believe that it is natural to expect the first-order approach itself to fail in such cases, since the agent's overall objective function will tend to be nonconcave [...]. The fact that CDFC and CISP imply concavity of the agent's payoff, regardless of the curvature of [the payment], then suggests that the CDFC/CISP conditions are very restrictive, not that the first-order approach is widely applicable.''%
}

Our main result, Theorem \ref{thm:main}, shows that the first-order approach is broadly valid as
long as the agent’s reservation utility is sufficiently high. The first-order approach holds even when
contracts are option-like and the agent’s problem has multiple local maxima. Theorem \ref{thm:main} also implies that the optimal contract exists, is unique, and is characterized by the standard simple formula. The main substantial condition is that the score of the output distribution is increasing in output.

The basic idea is simple, and clearly illustrated in basic examples as in Figure \ref{fig:gaussian-log-pp}. When reservation utility is very low -- say negative reservation wages -- the first-order approach often fails. Optimal contracts sometimes pay zero unless effort is quite high. This easily leads agents to be indifferent between equilibrium effort and giving up to nearly zero effort. However, as reservation utility increases, payoffs close to the intended action become more attractive, and the first-order approach becomes valid. Formally, the proof of Theorem \ref{thm:main} is built around the kink in the relaxed optimal contract moving far to the left, and this making the agent's problem more concave.

% Gaussian - log utility figure
\begin{figure}[p]
    \centering
    \includegraphics[width=\textwidth]{figures/log-gaussian-sigma=50.0/pp_stacked.pdf}
    \captionsetup{font=footnotesize} % Makes the note footnote-sized
    \caption{Optimal contracts with Gaussian distribution and log utility.}
    \label{fig:gaussian-log-pp}
    \caption*{\textit{Note:} Top panel: optimal wage function $w(y)$. Bottom panel: agent's expected utility $U(v^*, a)$ given optimal contract and action $a$. Colors represent reservation utility. Dashed lines indicate that the first-order approach is invalid at that reservation utility. Dots indicate the recommended action. The thin horizontal line indicates indifference between local maxima. Output has gaussian distribution with mean $a$ and standard deviation $50$, initial wealth is $50$ (both in thousands of dollars), and the cost function is $c(a) = a^2 / 30000$.}
\end{figure}

Section \ref{sec:examples} explains our results through simple examples and reconciles our findings with the literature. Section \ref{sec:model} gives definitions. Section \ref{sec:main-result} states Theorem \ref{thm:main}. Section \ref{sec:main-result-proof} outlines the proof. Section XXX provides a broad range of closed-form solutions for optimal contracts. Section XXX shows that optimal contracts are piecewise linear option contracts for log utility and output distributions in an exponential family with linear sufficient statistic (this has been independently discovered by \textcite{opp2025moral}). Section XXX provides numerical methods for both the case where the FOA is valid and in the more general case where it is not. Section XXX gives counter-examples and discusses limitations of our results. Section XXX discusses the relationship between our results and the literature.