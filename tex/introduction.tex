One of the workhorse models in economics and related fields is the principal-agent problem with moral hazard. The principal hires the agent to take an action $a$ in $\mathbb R _ +$ that affects the distribution of output $y$. The principal can only condition payments on realized output, and designs a contract $w(y)$ to provide incentives to the agent. The agent chooses the action $a$ to maximize her utility from wages minus her cost of effort,
\[
\mathbb{E} \left [ u(w(y)) | a \right ] - c(a)
\text{,}
\]
and the principal chooses the contract to maximize expected profits.

The main solution method for this problem is the first-order approach, which assumes that only local deviations in $a$ are binding. This yields a tractable, calculus-based formula for the optimal contract (see \cite{holmstrom1978incentives} and the excellent survey \cite{georgiadis2022contracting}). Many results in the literature rely on the first-order approach, even though \cite{mirrlees1999} (circulated in 1975) showed it is not always valid. In applied work, restrictive assumptions such as linear contracts or binary effort are often made to avoid this issue.%
\footnote{%
Examples of papers assuming the first-order approach include, to cite a few, \cite{jewitt2008moral, moroni2014existence, chaigneau2022should,castro2024disentangling}. Virtually all early work, including the seminal papers by \cite{holmstrom1978incentives, holmstrom1979moral}, and \cite{zeckhauser1970medical} assumes it. \cite{holmstrom1979moral} notes that 
``one has to assume that the agent's optimal choice of action is unique
for the optimal [...] This assumption seems very difficult to validate [...] and regrettably we have to leave the question about its validity open.'' \cite{edmans2009multiplicative} is an example of an important recent paper using both binary effort and linear contracts. See the surveys \cite{bolton2005,salanie2005economics,georgiadis2022contracting} for further references.
}

Unfortunately, existing sufficient conditions for the first-order approach are restrictive. The seminal papers are \cite{rogerson1985} and \cite{jewitt1988justifying}, followed by an extensive literature.%
\footnote{Important generalizations include \cite{sinclair1994first,conlon2009two,jung2015information,chade2020no}, and \cite{jung2024proxy}. \cite{chaigneau2022should, chaigneau2024theory} develop sufficient conditions with limited liability.}
\textcolor{blue}{Kadan, Reny, and Swinkels} (\citeyear{kadan2017existence}) summarize the general view that ``conditions facilitating the first-order approach are typically quite demanding''. The key issues are elegantly explained by \cite{chaigneau2022should} and \cite{conlon2009two}. They explain that the sufficient conditions require contracts to be close to concave, precluding common contracts such as compensation with stock options.% 
\footnote{\cite{conlon2009two} explains the intuition clearly: ``Unfortunately, the Jewitt conditions are tied to the concavity, not only of the technology, but also of the payment schedule. Thus, there are interesting cases where the Jewitt conditions should fail, since the payment schedule is often not concave. For example, managers often receive stock options, face liquidity constraints, [...]
I believe that it is natural to expect the first-order approach itself to fail in such cases, since the agent’s overall objective function will tend to be nonconcave [...]. The fact that CDFC and CISP imply concavity of the agent’s payoff, regardless of the curvature of [the payment], then suggests that the CDFC/CISP conditions are very restrictive, not that the first-order approach is widely applicable.''%
.}

Our main result, Theorem \ref{thm:main}, shows that the first-order approach is broadly valid, as long as the agent's reservation utility is sufficiently high. The result holds even when contracts are option-like and the agent's problem has multiple local maxima. We consider a setting with limited liability, because of its practical importance and because it is known that limited liability or a similar assumption is needed to guarantee existence \citep{moroni2014existence}. We find that, for low reservation utility, the standard view in the literature holds and the first-order approach often fails. However, for high reservation utility, we show that the first-order approach is satisfied. Thus, the issues with the validity of the first-order approach are likely to be important in examples with low reservation utility like when a monopsonist principal can impose highly unfavorable contracts on agents. When reservation utility is low, the first order approach is often invalid because it produces contracts where the agent has a global deviation to exert minimal effort. However, the first order approach is likely to be valid in settings with high reservation utility, such as competitive labor markets with high productivity workers. As reservation utility goes up, the first order approach's contract increases the incentive to work, eliminating global deviations, and making the first-order approach valid.

Theorem \ref{thm:main} also shows that the optimal contract exists, is unique, and characterized by a simple formula. Section \ref{sec:applications} show that optimal contracts are piecewise linear option contracts for many standard distributions and log agent utility. The theorem also shows that optimal contracts can be computed with simple algorithms. Thus, the limited liability model seems well suited for applied theory and empirical models that seek to predict the shape of optimal contracts $w(y)$.

Before going into the analysis, section \ref{sec:examples} explains our results through simple examples, and reconciles our findings with the literature. Section \ref{sec:model} gives definitions and section \ref{sec:main-result} states Theorem \ref{thm:main}. Section \ref{sec:main-result-proof} outlines the proof. Section \ref{sec:applications} considers applications, extensions, and counterexamples.