One of the workhorse models in economics is the principal-agent problem with moral hazard. The principal hires the agent to take an action $a$ in $\mathbb R _ +$ that affects the distribution $f(y|a)$ of output $y$. The principal can only condition payments on realized output, and designs a contract $w(y) \geq 0$ to provide incentives to the agent. The agent chooses the action $a$ to maximize her utility from wages minus her cost of effort $c(a)$ and the principal chooses the contract to maximize expected profits.

The main solution method for this problem is the first-order approach, which assumes that only local deviations in $a$ are binding, and yields a simple formula for the optimal contract.%
\footnote{%
See \textcite{holmstrom1978incentives}, the excellent survey \textcite{georgiadis2022contracting}, and recent work in \textcite{conlon2009two,kadan2017existence,chade2020no,chaigneau2022should,castro2024disentangling}.
}
Because of tractability, a vast literature simply assumes that the first-order approach is valid.%
\footnote{%
\textcite{mirrlees1999} showed that the first-order approach is not always valid. Examples of papers assuming the first-order approach include \textcite{jewitt2008moral, moroni2014existence, chaigneau2022should,castro2024disentangling}. Virtually all early work, including the seminal papers by \textcite{holmstrom1978incentives, holmstrom1979moral}, and \textcite{zeckhauser1970medical} assumes it. \textcite{holmstrom1979moral} notes that 
``one has to assume that the agent's optimal choice of action is unique
for the optimal [...] This assumption seems very difficult to validate [...] and regrettably we have to leave the question about its validity open.'' In applied work, restrictive assumptions such as linear contracts or binary effort are often made to avoid this issue. \textcite{edmans2009multiplicative} is an example of an important recent paper using both binary effort and linear contracts. See \textcite{bolton2005,salanie2005economics} for further references.
}


Unfortunately, existing sufficient conditions for the first-order approach to be valid are restrictive. The seminal papers are \textcite{rogerson1985} and \textcite{jewitt1988justifying}, followed by an extensive literature.%
\footnote{Important generalizations include \textcite{sinclair1994first,conlon2009two,jung2015information,chade2020no}, and \textcite{jung2024proxy}. \textcite{chaigneau2022should, chaigneau2024theory} develop sufficient conditions with limited liability.}
\textcite{kadan2017existence} summarize the general view that ``conditions facilitating the first-order approach are typically quite demanding.'' The key issues are elegantly explained by \textcite{chaigneau2022should} and \textcite{conlon2009two}. They explain that the sufficient conditions require contracts to be close to concave, precluding common contracts such as stock options.% 
\footnote{\textcite{conlon2009two} explains the intuition clearly: ``Unfortunately, the Jewitt conditions are tied to the concavity, not only of the technology, but also of the payment schedule. Thus, there are interesting cases where the Jewitt conditions should fail, since the payment schedule is often not concave. For example, managers often receive stock options, face liquidity constraints, [...]
I believe that it is natural to expect the first-order approach itself to fail in such cases, since the agent's overall objective function will tend to be nonconcave [...]. The fact that CDFC and CISP imply concavity of the agent's payoff, regardless of the curvature of [the payment], then suggests that the CDFC/CISP conditions are very restrictive, not that the first-order approach is widely applicable.''%
.}

Our main result, Theorem \ref{thm:main}, shows that the first-order approach is broadly valid as long as the agent's reservation utility is sufficiently high. This is vividly illustrated in numerical examples below (figures \ref{fig:gaussian-log-pp} to \ref{fig:poisson-log-pp}). In all these examples, the first-order approch is not valid for sufficiently negative reservation wages. This is consistent with the literature's view, that it is difficilt to guarantee that the first-order approach is valid for any reservation utility. However, in all the examples, the first-order approach is valid for any positive reservation wage. Thus, the first-order approach is broadly valid. The main assumptions in Theorem \ref{thm:main} are that the score is increasing and that there is limited liability, which is known to be important for existence \parencite{moroni2014existence}.

Theorem \ref{thm:main} also implies that the optimal contract exists, is unique, and is characterized by the standard simple formula. Section \ref{sec:applications} shows that optimal contracts are piecewise linear option contracts for many standard distributions and log agent utility. For many examples, the kink is far in the tail of the distribution of outcomes, so that contracts are effectively linear. This linearity result has been independently discovered by \textcite{opp2025moral}. Thus, the limited liability model seems well suited for applied theory and empirical models that seek to predict the shape of optimal contracts $w(y)$.

Before going into the analysis, section \ref{sec:examples} explains our results through simple examples and reconciles our findings with the literature. Section \ref{sec:model} gives definitions and section \ref{sec:main-result} states Theorem \ref{thm:main}. Section \ref{sec:main-result-proof} outlines the proof. Section \ref{sec:applications} considers applications, extensions, and counterexamples.