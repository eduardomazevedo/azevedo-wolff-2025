\subsection{Linear Contracts}
The formulae imply piecewise linear contracts in many examples. The key ingredients are log utility, which makes the wage function piecewise linear in the score, and a linear score function. This includes an exponential family with linear sufficient statistics. That is, when $f(y, a)$ is of the form
\begin{equation}
\label{eq:linear-exponential-family}
f(y | a)
=
\exp\left(\eta (a) y + A(a)\right)
\text{.}    
\end{equation}

We note this as follows:

\begin{remark}[Piecewise Linear Option Contracts]
    Assume that utility is log ($u(x) = \log(w_0 + x)$ with $w_0 > 0$) and that the distribution of outcomes is in an exponential family with a linear sufficient statistic (as in equation \ref{eq:linear-exponential-family}). Then the canonical contract wage function is piecewise linear. This includes the Gaussian, exponential, Poisson, geometric, binomial, and gamma distributions.
\end{remark}

Our piecewise linear contracts are option contracts. The contracts pay zero for low outcomes and increases linearly past a ``strike price.'' In some examples with a lower bound on the support, the optimal contract is linear (e.g., Figure \ref{fig:poisson-log-pp}). In other examples, optimal contracts are piecewise linear in some cases where the first-order approach is valid, but there are also low reservation utility cases where the first-order approach is invalid and optimal contracts are not linear (e.g., Figure \ref{fig:gaussian-log-pp}).
