We can now state our main result. We say that the first-order approach is valid for $(a_0, \bar U)$ if any solution to the relaxed cost minimization problem is a solution to the cost minimization problem.

Our main result is to show that the first-order approach is valid for sufficiently high reservation utility:

\begin{theorem}
    \label{thm:main}[Validity of the first order approach with high reservation utility]
    Given a regular intended action $a_0 > 0$, the first-order approach is valid for $(a_0, \bar U)$ for any sufficiently high $\bar U$.
\end{theorem}

The theorem formalizes the main point discussed in the introduction and examples. That the first order approach is valid under reasonable assumptions, as long as reservation utility is high.

The theorem simplifies the moral hazard problem because the relaxed problem is considerably more tractable. Proposition \ref{prop:relaxed-optimal-contract} shows that the relaxed problem has an almost everywhere unique solution given by a tractable formula. Combining theorem \ref{thm:main} and proposition \ref{prop:relaxed-optimal-contract} we obtain the following.

\begin{corollary}
    \label{thm:main}[Solution of the moral hazard problem with high reservation utility]
    Given a regular intended action $a_0 > 0$, there exist reservation utilities $U^*$ in $\mathbb R$ and $\bar U _R$ in $\mathbb R \cup \{\infty\}$ such that:
    \begin{enumerate}
        \item For $\bar U \geq \bar U _R$, neither the relaxed cost minimization problem nor the cost minimization problem are feasible.
        \item For $U^* \leq \bar U < \bar U _ R$, the cost minimization problem has an almost everywhere unique solution given by proposition \ref{prop:relaxed-optimal-contract}.
    \end{enumerate}
\end{corollary} 