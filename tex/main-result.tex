We say that \textbf{the first-order approach is valid} for an action and reservation utility pair $(a_0, \bar U)$ if any solution to the relaxed cost minimization problem is a solution to the cost minimization problem.

Our main result is that the first-order approach is valid for sufficiently high reservation utility:

\begin{theorem}
    \label{thm:main}[Validity of the first order approach with high reservation utility]
    Given an action $a_0 > 0$, the first-order approach is valid for $(a_0, \bar U)$ for any sufficiently high reservation utility $\bar U$.
\end{theorem}

The theorem formalizes the main point discussed in the introduction and examples. That the first order approach is valid under reasonable assumptions, as long as reservation utility is high. In particular, the theorem implies that the cost minimization problem has a tractable solution, given by the standard formulae in the first-order approach literature. We establish the solution in Proposition \ref{prop:relaxed-optimal-contract} below, and note the implication to the full problem here.

\begin{corollary}
    \label{cor:main}[Solution of the cost minimization problem]
    Given an action $a_0 > 0$, for any sufficiently high reservation utility $\bar U$, the cost minimization problem has an almost everywhere unique solution given by the formula in Proposition \ref{prop:relaxed-optimal-contract}.
\end{corollary}