\subsection{Discussion of the Literature}
\label{sec:literature-discussion}
\medskip\noindent\textit{\textbf{First Order Approach Literature.}}
There are three generations of contributions to the first-order approach. The first generation was started by \textcite{mirrlees1999}  (circulated in 1975), who showed that the first-order approach does not hold generally. Prior to \textcite{mirrlees1999}, the first-order approach was simply assumed with little justification. Subsequent papers became aware that they had to assume validity of the first-order approach \parencite{holmstrom1978incentives}, and developed results that do not depend on the first-order approach \parencite{grossman1983,araujo2001general}.

The seminal papers in the second generation are \textcite{rogerson1985} and \textcite{jewitt1988justifying}. They provided sufficient conditions for the first-order approach. Rogerson's conditions are well known to be strict, ruling out many natural distributions, including the Gaussian. Jewitt's conditions are more general, and include many interesting cases. Jewitt's main assumption is implicit, in that he requires his conditions to hold for all output levels, in particular ruling out limited liability. \textcite{chaigneau2022should} give a detailed explanation of how the Jewitt conditions are not applicable in the case of limited liability and survey the related literature. But limited liability is crucial to guarantee existence (see Section \ref{sec:counter-examples}). In many examples that satisfy the Jewitt conditions, an optimal contract does not exist. This includes the Gaussian - log utility example without limited liability. At face, value, the Jewitt conditions are satisfied. But the relaxed cost minimization problem does not have a solution. Thus, the first-order approach is ``valid'', since any of the non-existent solutions satisfies any property. But this validity is vacuous. This is similar to the colorful logical example that the statement ``all unicorns are pink'' is true, since unicorns do not exist. The second-generation papers have been influential, and much of the theoretical and applied work to this day assumes these conditions.

Our own results are inspired by \textcite{jewitt1988justifying}, and we build on his insight. We impose similar conditions on the part of the contract where limited liability does not bind, while allowing for limited liability. The proof of Theorem \ref{thm:main} shows that, with high reservation utility, the limited liability constraint binds with low probability. We then show, in similar lines to \textcite{jewitt1988justifying}, that the agent's problem becomes concave.

There is a third generation of work extending Jewitt's approach to multidimensional actions and richer information structures \parencite{conlon2009two,jung2015information,chade2020no,chaigneau2022should,jung2024proxy}. We believe our methods can be adapted to such environments, but we do not pursue that here.

At first glance, our Theorem \ref{thm:main} appears to contradict the standard view in the literature that the first order approach requires stringent conditions. In fact, the two are perfectly consistent. The reason is that the literature has looked for conditions under which the first-order approach is valid \emph{for every reservation utility}. Our examples show that indeed any such condition must rule out basic examples like the Gaussian-log utility example from Figure \ref{fig:gaussian-log-pp} -- and in fact all the numerical examples in this paper. Our contribution is to show that \emph{if the reservation utility is sufficiently high}, the first order approach is valid.


\medskip\noindent\textit{\textbf{Linear Contracts and Option Contracts.}}
A long literature seeks conditions under which optimal contracts are simple, such as linear contracts. The classic paper is \textcite{holmstrom1987aggregation}, who rationalize linear contracts with CARA utility in a continuous time dynamic model with Brownian output process. \textcite{innes1990limited} rationalizes piecewise linear contracts with a risk neutral agent, limited liability, and a monotonicity restriction on the wage function. Notable contributions include \textcite{chassang2013calibrated} and \textcite{carroll2015robustness} who rationalize linear contracts with risk neutral agents and a robust maxmin notion of optimality.

In contrast, we derive piecewise linear contracts under log utility and linear score. Our contracts are option contracts, with payoff of zero for low outcomes and linear increasing payment past a strike price. In some examples, our optimal contracts are linear in most of the support of the distribution of output. In some examples with a finite lower bound on the support, our optimal contracts are linear in the entire support.

The most closely related work is the contemporaneous paper \textcite{opp2025moral}. He independently derived the linearity result from remark \ref{rem:linear-contracts}. As far as we know, \textcite{opp2025moral} is the first to derive piecewise linear contracts under a broad set of conditions in standard moral hazard with risk aversion. \textcite{opp2025moral} includes a complete analysis of the exponential distribution case. He characterizes optimal contracts with and without limited liability and when the optimal exists without limited liability. In particular, he shows that the optimal contract with limited liability is globally linear for low intended action but has a kink for high intended action. We have no overlap with these results.

\textcite{hemmer1999introducing} assumes a number of Jewitt's conditions, and log utility with no limited liability. Their corollary 2 part 2 then shows that the optimal contract is globally linear. It is important to highlight the difference. They prove that contracts are globally linear, whereas we prove that contracts are piecewise linear. While their result is stronger, optimal contracts often do not exist in their setting (section \ref{sec:counter-examples}). Thus, \textcite{opp2025moral} and our paper are the first to derive piecewise linear contracts under a broad set of conditions, but \textcite{hemmer1999introducing} were the first to notice linearity of $k \circ g$ to the best of our knowledge. \textcite{chaigneau2015changes,chaigneau2017prudence} also attribute this observation to \textcite{hemmer1999introducing}.

Another closely related paper is \textcite{chaigneau2022should}. They consider the Gaussian-log utility case with limited liability, and assume validity of the first-order approach. Their lemma 2 shows that option contracts are optimal, as in our remark \ref{rem:linear-contracts}. This is the first result we know justifying option contracts in the sense of zero payment for low outcomes and linear increasing payment past a strike price. \citeauthor{holmstrom1978incentives}'s (\citeyear{holmstrom1978incentives}) thesis considers the case of log utility with an upper and lower bound on the wage function. He assumes the first-order approach and shows that the optimal contract is linear within those bounds.


\medskip\noindent\textit{\textbf{Solution of the Relaxed Cost Minimization Problem.}}
The solution to the relaxed cost minimization problem in proposition \ref{prop:relaxed-optimal-contract} is a trivial variation of the standard solution using the first-order approach. The case of limited liability is considered in \citeauthor{holmstrom1978incentives} (\citeyear{holmstrom1978incentives}, \citeyear{holmstrom1979moral}). \textcite{jewitt2008moral} consider an even more general case, with lower and upper bounds that depend on payments. Their elegant definition 1 is basically the definition of a canonical contract, and their proposition 1 basically the same as proposition \ref{prop:relaxed-optimal-contract}. We provide a proof of proposition \ref{prop:relaxed-optimal-contract} because our assumptions are different. \textcite{jewitt2008moral} assume the support of $y$ is a compact interval, the convexity of the distribution function condition, and bounded score. They argue that the bounded score is important for existence (p. 62), although in our setting solutions exist even with unbounded score. \textcite{moroni2014existence} corrects a mistake in \textcite{jewitt2008moral} showing the need of an assumption similar to utility being bounded below, which we make. Our existence proofs are different. \textcite{jewitt2008moral} prove that given a multiplier $\mu$ there exists a single multiplier $\lambda$ as a function of $\mu$ where the canonical contract satisfies the individual rationality constraint. They then use the intermediate value theorem to show that there exists a $\mu$ where the canonical contract satisfies all constraints. We instead prove (1) that the Lagrangian has a unique minimum that is a canonical contract (Lemma \ref{lem:lagrangian}), (2) that the Pareto problem has a solution that minimizes the Lagrangian (Lemma \ref{lem:pareto-problem}), and (3) that the Pareto problem solution solves the relaxed cost minimization problem (proof of Proposition \ref{prop:relaxed-optimal-contract}, Part 2). While we include the proof of proposition \ref{prop:relaxed-optimal-contract} for completeness, advanced readers may take it as trivial given the \citeauthor{holmstrom1978incentives} (\citeyear{holmstrom1978incentives}, \citeyear{holmstrom1979moral}), \textcite{jewitt2008moral}, and \textcite{moroni2014existence} results.


\medskip\noindent\textit{\textbf{High Reservation Utility.}}
In an influential paper, \textcite{chade2020highstakes} consider the principal-agent problem in the case where reservation utility converges to infinity. This paper is one of our inspirations in considering the case of high reservation utility, although it deals with completely different issues. They assume the first-order approach is valid (their assumption 4), and give no results on validity of the first-order approach. Their main result is that the wage cost to the principal is asymptotically the same as if effort was observable, so that agency costs are asymptotically small. This is a strong result that depends on detailed assumptions that rule out many of our examples. As they note, ``key assumptions driving our results are on the agent’s utility for income'' ... ``they are not without bite: log utility is excluded, and indeed the convergence results fail in this setting.''

We now discuss the connection in detail. The gist of their results is better understood heuristically in the CRRA case, where their result requires relative risk aversion $\gamma < 1$. Consider an agent with reservation wage $w_0$, which comprises most of the agent's wealth. Consider first perfect information. For the principal to compensate the agent for an effort cost $c$, he must pay the agent an additional of approximately $c / u'(w_0) \approx c w_0^\gamma$. Therefore, as $w_0$ converges to infinity, the additional incentive pay is a vanishingly small fraction of the reservation wage, as long as $\gamma < 1$. Their analysis formalizes how, in this case, the incentive pay is also asymptotically small with moral hazard. And this makes clear why the result depends on $\gamma < 1$ and thus rules out log utility.

Their actual analysis is based on an elegant approximation.%
\footnote{We were influenced by their approximation, but for our analysis it is more convenient to consider the average slope of $g$ over long spans of $y$ (see appendix \ref{sec:appendix-proofs}).}
The key idea is to approximate the optimal contract as a linear function of the score:
\[
    v(y)
    =
    g\bigg(\lambda + \mu S(y|a_0)\bigg)
    \approx
    g(\lambda) + g'(\lambda) \mu S(y|a_0)
    \text{.}
\]
Integrating the (\ref{IR}) and (\ref{GIC}) constraints yields (as in their proposition 1)
\[
    g(\lambda) \approx c(a_0) + \bar U
    \text{,}
\]
\[
    \mu g'(\lambda) \int S^2(y|a_0) f(y|a_0) \, dy \approx c'(a_0)
    \text{.}
\]
Substituting the CRRA case in the second equation implies that $\mu \lambda ^ {-\gamma}$ is approximately constant, and thus $\mu / \lambda$ converges to zero in the case $\gamma < 1$. Their paper formalizes this point and gives the exact result. While we draw on their insight of considering high reservation utility, this discussion clarifies the differences and our marginal contribution. First, they assume the first-order approach is valid and give no results on validity of the first-order approach. Second, they make different and strong assumptions that rule out much of our analysis. Third, our results broadly hold in many settings where their asymptotics do not. For example, their results show that agency costs are asymptotically small compared to reservation wages. In all of the numerical examples in this paper, agency costs are significant, nevertheless we see that the first-order approach is broadly valid.

\textcite{castro2024disentangling} assumes that the principal's expected wage cost is convex in the intended action, which they motivate with a result by \textcite{chade2020highstakes} that this is true with high stakes. But \textcite{castro2024disentangling} does not otherwise use high reservation utility. They assume that the first-order approach is valid (their assumption 1). Neither paper assumes limited liability, and \textcite{castro2024disentangling} assumes that utility in unbounded below but uses a bounded below utility in their numerical results. \textcite{chade2025jeroen} consider a moral hazard problem with a set actions in an interval plus one non-ordered action. They consider the relaxed problem, assuming that the first-order approach is valid among the actions in the interval. They consider high stakes using the results in \textcite{chade2020highstakes} to show that the optimal contract with several different utility functions converges to essentially the same contract.

\medskip\noindent\textit{\textbf{Numerical Analysis.}}
There are four general approaches to numerically solve the moral hazard problem without resorting to the first-order approach.

The first approach is to include additional no-jump constraints in the relaxed cost minimization problem. This was introduced in the 1970s by \textcite{mirrlees1999} and improved by \textcite{araujo2001general}. The main limitation is that this relies on ad hoc tricks to decide which constraints to add (see \textcite{ke2018general} p. 1426). Motivated by this, \textcite{ke2018general} introduce a sandwich method where the additional constraint is added adversarially. This method works well in examples where part of the problem happens to be analytically tractable.

The two most popular approaches start by discretizing the set of actions, so that there is only a finite number of global incentive compatibility constraints (following \textcite{grossman1983}). The linear programming approach maximizes over a joint distribution of actions, output levels, and utility from wages, all of which have to be in a discrete grid \parencite{prescott1999primer, su2007computation}. While this has the advantage of making the principal's problem a linear program, the main difficulty is the curse of dimensionality: the number of variables is the product of the sizes of the three grids. To address this, \textcite{su2007computation} introduced the mathematical program with equilibrium constraints approach. This approach maximizes over the contract and a distribution over actions. So the number of variables is the number of possible output values plus the number of possible actions. These approaches are often applied to relatively small problems. For example, \textcite{su2007computation}'s main numerical example has two possible output levels.

The ideal paper to contrast our algorithm with these existing methods is \textcite{armstrong2007stock}. They model executive compensation with log normally distributed output. This is exactly the same signal structure as our examples with the Gaussian distribution, but with the principal payoff being an exponential function of the action. They use mathematical programming with equilibrium constraints. Their figure 3 displays some example optimal contracts. The figure displays some of the key features of optimal contracts, such as a region of zero payment followed by an increasing region following the appropriate functional form. These are impressive results far ahead of their time. However, their algorithm does not match closely the optimal contract far away from the support of the distribution of output. Moreover, they report that each solution to the principal's problem takes several hours. The same problems take about 1E-1 seconds in our algorithm.

The fourth approach in the literature is to approximate the agent's utility as a function of actions with a ratio of polynomials \parencite{renner2015polynomial}. \textcite{renner2015polynomial} then show that the principal's problem can be reduced to a nonlinear program. Crucially, their method can give global optimality guarantees. They illustrate their method solving the log normal example from \textcite{armstrong2007stock}.

In terms of algorithms, our approach is most related to the no-jump approach. We simply use a reasonable heuristic to add constraints, while taking advantage of Lagrange duality, parametric assumptions for fast dual calculations with analytic gradients, caching, and warm starts. Our algorithm has the disadvantage of not giving a global optimality guarantee, and is thus complementary to the LP and polynomial approaches. For example, one could use our algorithm to obtain a solution candidate and check it with the LP or polynomial approaches.