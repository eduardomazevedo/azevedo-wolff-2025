The proof of the main result follows from two key propositions. Proposition \ref{prop:relaxed-optimal-contract} characterizes the solution of the relaxed problem. Proposition \ref{prop:concave} shows that, for sufficiently high reservation utility, the agent's problem is concave. We now state these propositions and explain the key steps in the argument. Appendix \ref{sec:appendix-proofs} contains the proofs.

For the relaxed problem to be well-defined, the derivative $\partial _a U(v,a)$ must exist. Remark \ref{rem:leibniz} in the appendix shows that this is true under our assumptions, and moreover, that we can calculate the derivative by differentiating under the integral sign. Henceforth, we will use differentiation under the integral sign without referencing Remark \ref{rem:leibniz}.

Throughout this section, fix a regular intended action $a_0$. To simplify notation, we omit the dependence on $a_0$ whenever it is clear, writing, for example, $U(v)$ instead of $U(v, a_0)$.

\subsection{Solution to the Relaxed Problem}

We first show that the relaxed cost minimization problem has an almost everywhere unique solution, and that this solution has a simple formula in terms of Lagrange multipliers. The solution is based on the first-order condition and is similar to standard formulas in the literature.\footnote{%
This type of formula for the optimal contract is well-known and central in the literature that assumes the first-order approach. The seminal reference is \textcite{holmstrom1978incentives}, and recent examples with limited liability include \textcite{jewitt2008moral} and \textcite{chaigneau2022should}. The survey \textcite{georgiadis2022contracting} has an excellent explanation of the formula in its first paragraphs.}

The relaxed cost minimization problem is convex. Define its Lagrangian as
\begin{equation}
    \label{eq:lagrangian}
    \mathcal{L}(v,\lambda,\mu):=W(v)+\lambda\left(\bar{U}-U(v)\right)+\mu(-U_{a}(v))\text{.}
\end{equation}

Heuristically differentiating this Lagrangian with respect to $v(y)$ and setting the derivative to zero gives
\[
k'(v(y)) f(y | a) = \lambda f(y | a) + \mu f_{a}(y|a_{0})\text{.}
\]

Dividing by $f(y | a)$ gives
\begin{equation}
\label{eq:foc}
\tag{FOC}
k'(v(y)) = \lambda + \mu \frac{f_{a}(y|a_{0})}{f(y|a_{0})}\text{.}
\end{equation}

Equation (\ref{eq:foc}) is the key step in the standard first-order approach literature. Appendix \ref{sec:appendix-proofs} formally analyzes the convex program, demonstrates existence and uniqueness of the solution, and characterizes the optimal contract based on equation (\ref{eq:foc}).

In our case of limited liability, the solution is described clearly with the following notation. Define the {\bf optimal expected wage} $\omega(\bar U)$ as the value of the infimum in the relaxed cost minimization problem. Define the \textbf{link function} $g:\mathbb R \rightarrow \mathbb R$ as\footnote{%
The link function's input, $z$, is a marginal dollar cost of providing one util to the agent  (measured in units of $\frac{\text{\$}}{\text{util}} $). The link function evaluated at $z$, $g(z)$, returns the utility level where $z$ is the marginal cost of utility to the agent. Any $z$ below $\frac{1}{u'(0)}$, which is the cheapest possible marginal cost, returns $g(z) = u(0)$.
}
\[
    g(z):=k'^{-1}\left(\max\left\{ \frac{1}{u'(0)},z\right\} \right) \text{.}
\]

Define the \textbf{score} function\footnote{The score is known as the likelihood ratio in the economics literature. We favor the term score to be in line with the modern scientific literature.} as in the statistics literature:
\[
    S(y | a) := \frac{\partial}{\partial a} \log f(y|a)
    =
    \frac{f_a(y|a)}{f(y|a)}
    \text{.}
\]
 
\begin{definition}
    \label{def:canonical-contract}
    A canonical contract $V(y | \lambda, \mu)$ is defined for $\lambda$ and $\mu$ in $\mathbb R$ as
    \begin{equation}
    \label{eq:canonical-contract}
    V\left(y | \lambda, \mu \right)
    :=
    g \left(\lambda + \mu S(y|a_0) \right) \text{.}
    \end{equation}
\end{definition}

The following proposition shows that the relaxed problem has a solution and characterizes the solution and Pareto frontier.

\begin{proposition}
    \label{prop:relaxed-optimal-contract}
    [Solution of the Relaxed Cost-Minimization Problem]
    There exists $\bar U_L$ in $\mathbb R$ such that:
    \begin{enumerate}
        \item (existence, uniqueness, and characterization) The relaxed problem has an almost everywhere unique solution $v^*(y|\bar U)$. There exist Lagrange multipliers $\lambda^*(\bar U) \geq 0$ and $\mu^*(\bar U) > 0$ such that the solution is almost everywhere equal to the canonical contract
    \[
        \label{eq:relaxed-optimal-contract}
        v^*(y|\bar U) := V\left(y | \lambda^*(\bar U), \mu^*(\bar U)\right)
        \text{.}
    \]
        \item (comparative statics)
        \begin{itemize}
            \item For $\bar U \leq \bar U_L$, we have $\lambda^*(\bar U) = 0$. The relaxed optimal contract $v^*(\cdot , \bar U)$ and optimal expected wage $\omega(\bar U)$ do not vary with $\bar U$ in this range.
            \item For $\bar U > \bar U_L$, $\lambda^*(\bar U)$ is strictly increasing and $\lim_{\bar U \rightarrow \infty} \lambda^*(\bar U) = \infty$. The optimal expected wage $\omega(\bar U)$ is strictly increasing and strictly convex.
        \end{itemize}
    \end{enumerate}
\end{proposition}

The proposition shows that the solution to the relaxed cost minimization problem is unique and given by a canonical contract. Moreover, the Pareto frontier of profits and agent utility is convex.

The Gaussian-log utility example illustrates proposition \ref{prop:relaxed-optimal-contract}. The score is $(y-a) / \sigma^2$, the compensation cost function is $k(v)=\exp (v) - w_0$, and the link function is $\log \max \{z, w_0\}$. This implies the wage function
\[
 k(v^*(y|\bar U))
 =
 \left[\lambda^*(\bar U) + \mu^*(\bar U) \frac{y-a}{\sigma^2} - w_0 \right] ^ +\text{.}
\]
This is the piecewise linear solution that we saw numerically in figure \ref{fig:norm_log_con_opt}. The convex Pareto frontier is illustrated in figure \ref{fig:pf}.

The figure illustrates the comparative statics from part (2). For sufficiently low $\bar U$, the (\ref{IR}) constraint is not binding, and thus the solution does not depend on $\bar U$. In the region where the (\ref{IR}) constraint is binding, expected wages and the Lagrange multiplier $\lambda^*(\bar U)$ are strictly increasing in $\bar U$.

% Pareto frontier figure
\begin{figure}[ht]
    \centering
    \includegraphics[width=0.8\textwidth]{figures/norm_log_pf.pdf}
    \caption{Pareto Frontier with Gaussian Distribution and Log Utility}
    \label{fig:pf}
    \captionsetup{font=footnotesize} % Makes the note footnote-sized
    \caption*{\textit{Note:} This figure displays the relaxed problem's Pareto frontier. The agent's utility is $u(x) = log(\frac{x}{20K})$, the cost function is $c(a) = \theta a^2$, where $\theta = .01$, and the agent's starting wealth is $w_0 = 50K$. The output is $y \sim N(a,(20K^2))$, where $a$ is the agent's action. The intended action is $a_0 = 100K$.}
\end{figure}

The formula in proposition \ref{prop:relaxed-optimal-contract} is essentially the standard first-order approach's equation (\ref{eq:foc}), but properly accounts for limited liability. The intuition is the following: the principal would like to always pay the agent $g(\lambda ^* (\bar U))$, which is the utility level where the marginal cost of providing utility to the agent is $\lambda ^* (\bar U)$. However, incentive compatibility requires that payment depends on whether there is statistical evidence of high effort. The optimal contract pays more if the score is positive and less if the score is negative.

Proposition \ref{prop:relaxed-optimal-contract} deviates substantially from the literature without limited liability. Our formula for the optimal contract is different because it gives the agent a constant payment of $0$ for sufficiently low outcomes, where the limited liability constraint binds (see the kink in the wage function in figure (\ref{fig:norm_log_con})). Proposition \ref{prop:relaxed-optimal-contract} guarantees existence of an optimum in the relaxed problem, whereas without limited liability the relaxed problem often has no solution. This stark difference can be seen in the Gaussian-log utility example. Without limited liability, any contract satisfying the first-order conditions would pay a {\em globally} linear wage. But this would imply negative consumption in some states, and thus $-\infty$ utility to the agent. Indeed, the Gaussian-log utility example illustrates \textcite{mirrlees1999}'s classic point that optimal contracts may not exist without limited liability.

\subsection{High Reservation Utility} 

We now demonstrate that, for sufficiently high reservation utility, the solution to the relaxed cost minimization problem also solves the original cost minimization problem. To do so, we show that the relaxed problem's solution satisfies the global incentive compatibility constraint. This is achieved by demonstrating the stronger result that the agent's utility $U(v^*,a)$ is concave in $a$ at the relaxed optimal contract.

\begin{proposition} 
    \label{prop:concave} 
    [Concavity of the Agent's Problem for High Reservation Utility]
    There exists $U^*$ in $\mathbb{R}$ such that, for all $\bar U \geq U^*$ and all $a \in \mathcal{A}$,  
    $$ U_{aa}(v^* (y | \bar{U}), a) \leq 0 .$$ 
\end{proposition}

Theorem \ref{thm:main} is a direct consequence of this fact:

\begin{proof}[Proof of Theorem \ref{thm:main}]
    Take $\bar U \geq U ^*$. If the relaxed cost minimization problem has a solution $v^*(y | \bar U)$, proposition \ref{prop:concave} implies that the agent's expected utility $U(v^*,a)$ is concave in $a$. Therefore, the global incentive compatibility constraint is satisfied, and the solution to the relaxed cost minimization problem is also a solution to the cost minimization problem.
\end{proof} 

Proposition \ref{prop:concave} shows that high reservation utility is crucial for the first-order approach's validity. Figure (\ref{fig:norm_log}) helps illustrate this result by plotting the relaxed optimal contract and the agent's expected utility versus her effort under the relaxed-optimal contract in the Gaussian-log utility example. The charcoal line represents the expected utility function for the relaxed-optimal contract with the lowest reservation utility, $\bar U = u(0)$. When reservation utility is low, the agent's problem is not concave, and choosing an action close to $0$ is a profitable global deviation for the agent. The limited liability constraint makes low effort appealing for the agent; it ensures that the agent can always achieve utility strictly greater than $u(0)$ by exerting no effort. If the (\ref{IR_relaxed}) constraint binds, as it does in many of our examples, then the agent's expected utility from choosing the intended action is $\bar U$, and the first-order approach surely fails when $\bar U = u(0)$.

\begin{figure}[ht]
    \centering
    \begin{subfigure}[b]{0.45\textwidth}
        \includegraphics[width=\textwidth]{figures/norm_log_con.pdf}
        \caption{Canonical Contracts}
        \label{fig:norm_log_con}
    \end{subfigure}
    \hfill
    \begin{subfigure}[b]{0.45\textwidth}
        \includegraphics[width=\textwidth]{figures/norm_log_util.pdf}
        \caption{Expected Utility}
        \label{fig:norm_log_util}
    \end{subfigure}
    \caption{Relaxed Optimal Contracts with Gaussian Distribution and log Utility}
    \label{fig:fig1}
    \captionsetup{font=footnotesize} % Makes the note footnote-sized
    \caption*{\textit{Note:} Panel a displays the wage function that solves the relaxed problem at different values of $\bar U$, and Panel b displays the agent's expected utility from a grid of possible actions given the corresponding wage functions from Panel a. The values of $\bar U$ are expressed in terms of certainty equivalents. The agent's utility is $u(x) = \log(\frac{x}{20K})$, the cost function is $c(a) = \theta a^2$, where $\theta = .01$, and the agent's starting wealth is $w_0 = 50K$. The output is $y \sim N(a,(20K^2))$, where $a$ is the agent's action. The intended action is $a_0 = 100K$.}
    \label{fig:norm_log}
\end{figure}

Figure (\ref{fig:norm_log}) also illustrates how, as reservation utility increases, the first-order approach becomes valid. The (\ref{IR_relaxed}) constraint ensures that $U(v^*,a_0)$ must be greater than $\bar U$. The rightmost local maximum, $U(v^*,a_0)$, therefore, increases as $\bar U$ increases. The leftmost local maximum also rises, but much more slowly. In the example, the first-order approach becomes valid with a very small increase in $\bar U$ above $u(0)$, as the rightmost local maximum overtakes the leftmost local maximum. Concavity of $U(v^*,a)$ on the other hand, usually requires much larger increases to $\bar U$, so, there are many values of $\bar U$ where the first-order approach is valid, even though $U(v^*,a)$ is not concave in $a$.

We now clarify the intuition of why the agent's problem is concave for high reservation utility, but not for low reservation utility. The relaxed optimal contract has two regions. When the outcome $y$ is below a threshold, the limited liability constraint binds, and the contract specifies a constant wage of $0$. When $y$ is above the threshold, the agent's payment is dictated by equation (\ref{eq:foc}). This kink introduces convexity into the agent's problem.

The key observation is that, as reservation utility increases, the kink tends to move to the left, so that the agent is paid with high probability. This is illustrated in figure \ref{fig:norm_log_con}. Thus, as reservation utility increases, the kink becomes less relevant, and the agent's problem becomes more concave due to the curvature of the cost function.

\begin{figure}[ht]
    \centering
    \begin{subfigure}[b]{0.45\textwidth}
        \includegraphics[width=\textwidth]{figures/norm_log_probs.pdf}
        \caption{Probability of Payment}
        \label{fig:norm_log_probs}
    \end{subfigure}
    \hfill
    \begin{subfigure}[b]{0.45\textwidth}
        \includegraphics[width=\textwidth]{figures/norm_log_exp_wage.pdf}
        \caption{Expected Wage}
        \label{fig:norm_log_exp_wage}
    \end{subfigure}
    \caption{Probability of Payment and Expected Wages in Gaussian-Log Utility Example}
    \captionsetup{font=footnotesize} % Makes the note footnote-sized
    \caption*{\textit{Note:} Panel A plots the probability that the agent receives a strictly positive wage against the agent's action, under the relaxed optimal contract for different values of reservation utility. Panel B plots the expected wage against effort for the same contracts. The parameters of the problem are as in figure \ref{fig:fig1}.}
\end{figure}

Appendix \ref{sec:appendix-proofs} formalizes this argument in the proof of proposition \ref{prop:concave}. The proof has two key steps. First, it is shown that the probability that the agent receives zero payment converges to $0$. The intuition is that, when $\bar U$ is large, the typical payment received by the agent is large. So, if the agent could have a high impact on the probability of receiving a non-zero payment, she would have incentives to work harder than the intended effort level. This would violate the first-order condition, so the probability of receiving zero payment must converge to $0$ (lemma \ref{lem:threshold-outcome}).

The second key step is to show that this implies that the agent's expected utility is concave in $a$. The intuition is that the kink is very far to the left. Moreover, most of the increase in utility going from a payment of zero to the typical payment happens close to the kink. The proof shows that, due to this, the curvature of expected utility is dominated by the curvature of the cost function (lemma \ref{lem:inf_lambda_second_deriv}).