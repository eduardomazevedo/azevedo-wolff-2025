Consider the following \textbf{Gaussian-log utility example}. A risk-neutral principal hires an agent to work. The agent chooses action $a$ at a cost $c(a)$ proportional to $a^2$. Output $y$, which accrues to the principal, is normally distributed with mean $a$ and standard deviation $\sigma$. The agent has some independent assets and log utility, so that her utility from a payment of $x \geq 0$ is $u(x) = \log(w_0 + x)$. For concreteness, suppose the principal is the owner of a small company and the agent is a professional manager. We consider $a$ choices in the order of \$100{,}000, standard deviation $\sigma$ of \$50{,}000, and $w_0$ equal to \$50{,}000. The agent has reservation wage $\bar w$.

The principal designs a compensation contract with wage $w(y) \geq 0$ to maximize profits. The optimal contract balances the goals of inducing effort, reducing the agent's risk, and providing the agent with her reservation utility.

Figure \ref{fig:gaussian-log-pp} plots optimal contracts for a range of reservation wages $\bar w$, with certainty equivalents from \$0 to \$50,000. The top panel plots the optimal wage schedules $w(y)$. The bottom panel plots the agent's expected utility $U(v^*, a)$ of an optimal contract $v^*$ as a function of action $a$. The intended action is denoted by the red dot in the bottom panel.

The bottom panel shows when cases the first-order approach is valid. The first-order approach fails whenever there is a binding non-local deviation. In these cases, solving the problem with only the local incentive constraint leads to incorrect solutions. Cases in which the first-order approach fails are denoted with dashed lines.

We see that the first-order approach is valid for all weakly positive reservation wages, and only fails for sufficiently negative reservation wages. This is an illustration of our main point, that the first-order approach is broadly valid in most interesting cases. Moreover, this is representative of a more general phenomenon, and not simply this example. Figures \ref{fig:poisson-log-pp} and \ref{fig:gaussian-cara-pp} consider a different distribution (the discrete Poisson distribution) and a different utility function (CARA). The same pattern holds.

% Poisson - log utility figure
\thispagestyle{empty}
\begin{figure}[p]
    \centering
    \includegraphics[width=\textwidth]{figures/log-poisson/pp_stacked.pdf}
    \captionsetup{font=footnotesize} % Makes the note footnote-sized
    \caption{Optimal contracts with Poisson distribution and log utility.}
    \label{fig:poisson-log-pp}
    \caption*{\textit{Note:} Top panel: optimal wage function $w(y)$. Bottom panel: agent's expected utility $U(v^*, a)$ Colors represent reservation utility. Dashed lines indicate that the first-order approach is invalid at that reservation utility. Bottom panel dots indicate the recommended action. The thin horizontal line indicates indifference between local maxima. Output has Poisson distribution with mean $a$, initial wealth is $50$ (both in thousands of dollars), and the cost function is $c(a) = a^2 / 2800$.}
\end{figure}


% Gaussian - CARA utility figure
\thispagestyle{empty}
\begin{figure}[p]
    \centering
    \includegraphics[width=\textwidth]{figures/cara-gaussian-sigma=50.0/pp_stacked.pdf}
    \captionsetup{font=footnotesize} % Makes the note footnote-sized
    \caption{Optimal contracts with Gaussian distribution and CARA utility.}
    \label{fig:gaussian-cara-pp}
    \caption*{\textit{Note:} Top panel: optimal wage function $w(y)$. Bottom panel: agent's expected utility $U(v^*, a)$ Colors represent reservation utility. Dashed lines indicate that the first-order approach is invalid at that reservation utility. Bottom panel dots indicate the recommended action. The thin horizontal line indicates indifference between local maxima. arameters are as in figure \ref{fig:gaussian-log-pp}, but with CARA coefficient set to relative risk aversion of $1$ at the initial wealth.}
\end{figure}

Theorem \ref{thm:main} demonstrates that these many examples are no coincidence. The theorem formally demonstrates that the first-order approach is valid for sufficiently high reservation utility. The examples illustrate the idea in simple terms. For very low reservation utility, the first-order approach fails. The reason is that principal offers tough contracts, with zero pay for low output. Optimal contracts have a kink, and only start paying a positive wage relatively close to the intended action. Thus, the agent has a global deviation to basically give up and exert no effort. Thus, the first-order approach fails with very low reservation utility.

However, as \emph{reservation utility increases, the first-order approach becomes valid}. The reason is that the principal is forced to offer contracts with higher utility at the recommended action. This constraint moves the rightmost local maximum of $U(v^*, a)$ up. The rightmost maximum goes up faster than the leftmost local maximum. Thus, the rightmost maximum becomes the unique global maximum. In the examples, the first-order approach is valid for any positive reservation wage. The proof of Theorem \ref{thm:main} works with a somewhat different logic. The key observation is that the kink in the optimal contract moves to the left, and this makes the agent's problem more concave. This then implies that the agent's problem becomes more concave, and the first-order approach becomes valid.

Theorem \ref{thm:main} does not contradict existing results in the literature. The low reservation utility case shows that the literature consensus that it is hard to guarantee that the first-order approach is valid for any reservation utility is correct -- \emph{if we insist that the first-order approach is valid for all reservation utilities}. We see that, even in the Gaussian-log utility case, the first-order approach fails. Thus, any conditions guaranteeing the first-order approach are valid for all reservation utilities must be quite restrictive. The examples formally demonstrate the point made by \textcite{chaigneau2022should} and \textcite{conlon2009two}, that requiring the first-order approach to be valid for all reservation utilities excludes many interesting cases.




The first-order approach fails at the lowest reservation utility, with certainty equivalent of -\$1,000. The optimal contract is a tough bargain. The agent is expected to exert effort $a$ equal to about \$$130,000K$. The contract $w(y)$ specifies a payment of zero for any output lower than about \$$100,000$. Thus, the agent needs to produce relatively high output to get paid at all. This contract induces two local maxima in the function $U(v^*, a)$. The agent is indifferent between essentially giving up with a local maximum at $a \approx 0$ and a local maximum at the recommended effort level $a \approx \$130,000$. This is consistent with the standard view in the literature. Option-like contracts are natural, and thus it is difficult to find general sufficient conditions under which the first-order approach holds. In particular, it is impossible to find sufficient conditions for the first-order approach that include the Gaussian-log utility example, and that work regardless of reservation utility.

Surprisingly, the first-order approach becomes valid as soon as reservation utility is slightly higher. The constraint of giving the agent higher utility forces the principal to offer contracts with a higher value of $U(v^*, a)$ at the recommended action. This constraint moves the rightmost local maximum of $U(v^*, a)$ up. The rightmost maximum goes up faster than the leftmost local maximum. Thus, the rightmost maximum becomes the unique global maximum. Numerically, the first-order approach is valid for any positive reservation wage.

Theorem \ref{thm:main} shows that this example is representative of a more general phenomenon. The theorem shows that, under certain conditions, the first-order approach is always valid for sufficiently high reservation utility. The most substantial condition is that the score is increasing in output (monotone likelihood ratio). That is, greater output is evidence of greater effort. Figures \ref{fig:grid_contract_opt} and \ref{fig:grid_util_opt} illustrate the generality of this point by plotting examples with different risk preferences and output distributions. All the examples display the same qualitative pattern, consistent with Theorem \ref{thm:main}. The monotone score assumption is crucial. For example, if output has a fat-tailed Student-$t$ distribution, there are examples where the first-order approach fails even for high reservation utilities. The reason is that positive and negative outliers are not very informative about effort, so the agent is not penalized for very poor outcomes, which generates global deviations to low effort.


Theorem \ref{thm:main} also guarantees existence and uniqueness of an optimal contract. Moreover, the solution has a simple calculus-based formula, as in the classic first-order approach literature following \textcite{holmstrom1978incentives}. The theorem imlpies that, beause we can use the first-order approach, the optimal wage equals%
\footnote{%
These additional results are simple given the validity of the first-order approach. Our analysis and results are close to those in \textcite{jewitt2008moral}, albeit we cannot simply use their results due to slightly different assumptions. For example, they assume the \textcite{rogerson1985} conditions, which rule out examples such as the Gaussian distribution of output. Hence, we cannot simply use their results, although our proofs follow broadly similar lines.}
\begin{equation}
    \label{eq:optimal-wage}
    w(y) = k \circ g\biggl(\mu + \lambda S(y | a_0)\biggr) \text{.}
\end{equation}

The function $k \circ g$ is determined by the utility function and the limited liability constraint. $S(y | a_0)$ is the score function (also known as the likelihood ratio), determined by the distribution of output. The optimal action is $a_0$, and $\lambda$ and $\mu$ are Lagrange multipliers.

Theorem \ref{thm:main} makes our limited liability setting particularly amenable to applications. As long as one considers a competitive setting with sufficiently high reservation utility, the main difficulties are resolved. Optimal contracts exist, are unique, and can be calculated trivially from equation (\ref{eq:optimal-wage}).

Section \ref{sec:applications} provides results to help researchers use the limited liability model. We provide simple calculus formulae for $k \circ g$ and $S$ covering a wide set of examples. These can be readily applied in both theoretical and numerical models. We provide accompanying code for numerical solutions.

An immediate corollary of equation (\ref{eq:optimal-wage}) is that piecewise linear contracts are optimal in a number of examples. In the case of log utility, $k \circ g$ is piecewise linear. For many output distributions, including the Gaussian, the score is linear. This holds for any distribution in the exponential family with linear sufficient statistic. This includes the Gaussian, exponential, Poisson, and gamma distributions. This is illustrated in figures XXX and XXX, where all the contracts where the first-order approach is valid are piecewise linear (although the rare contracts where the first-order approach is not valid are not). This linearity result has been independently discovered by \textcite{opp2025moral}, who also established many results not included here, especially for the case of the exponential distribution.

Piecewise linear contracts are options, with a constant value up to a point, and increasing thereafter. Our finding extends previous results that rationalize option-like contracts \textcite{innes1990limited, jewitt2008moral, chaigneau2024theory} and contributes to the literature rationalizing linear contracts \textcite{holmstrom1987aggregation,carroll2015robustness}.