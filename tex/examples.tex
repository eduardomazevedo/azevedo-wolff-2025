Consider the following \textbf{Gaussian-log utility example}. A risk-neutral principal hires an agent to perform a task. The agent chooses action $a$ at a cost $c(a)$ proportional to $a^2$. Output $y$, which accrues to the principal, is normally distributed with mean $a$ and standard deviation $\sigma$. The agent has some initial wealth and log utility, so that her utility from a payment of $x \geq 0$ is $u(x) = \log(w_0 + x)$. We consider $a$ choices in the order of \$100{,}000, standard deviation $\sigma$ of \$50{,}000, and $w_0$ equal to \$50{,}000. The agent has reservation wage $\bar w$.

The principal designs a compensation contract with wage $w(y) \geq 0$ to maximize profits. The optimal contract balances the goals of inducing effort, reducing the agent's risk, and providing the agent with her reservation utility.

Figure \ref{fig:gaussian-log-pp} plots optimal contracts for a range of reservation wages $\bar w$, with certainty equivalents from \$0 to \$50,000. The top panel plots the optimal wage schedules $w(y)$. The bottom panel plots the agent's expected utility $U(v^*, a)$ of an optimal contract $v^*$ as a function of action $a$. The intended action is denoted by the red dot in the bottom panel.

The bottom panel shows for what reservation wages the first-order approach is valid. The first-order approach fails whenever there is a binding non-local deviation. In these cases, solving the problem with only the local incentive constraint leads to incorrect solutions. Cases in which the first-order approach fails are denoted with dashed lines.

The first-order approach fails with a reservation wage of $-\$1,000$, but is valid for all the illustrated positive reservation wages. This is an illustration of our main point, that the first-order approach is broadly valid in most interesting cases. Moreover, this is representative of a more general phenomenon, and not simply this example. Figures \ref{fig:poisson-log-pp} and \ref{fig:gaussian-cara-pp} consider a different distribution (the discrete Poisson distribution) and a different utility function (CARA). The same pattern holds, as it does in a multitude of other examples.

% Poisson - log utility figure
\thispagestyle{empty}
\begin{figure}[p]
    \centering
    \includegraphics[width=\textwidth]{figures/log-poisson/pp_stacked.pdf}
    \captionsetup{font=footnotesize} % Makes the note footnote-sized
    \caption{Optimal contracts with Poisson distribution and log utility.}
    \label{fig:poisson-log-pp}
    \caption*{\textit{Note:} Top panel: optimal wage function $w(y)$. Bottom panel: agent's expected utility $U(v^*, a)$ Colors represent reservation utility. Dashed lines indicate that the first-order approach is invalid at that reservation utility. Bottom panel dots indicate the recommended action. The thin horizontal line indicates indifference between local maxima. Output has Poisson distribution with mean $a$, initial wealth is $50$ (both in thousands of dollars), and the cost function is $c(a) = a^2 / 2800$.}
\end{figure}


% Gaussian - CARA utility figure
\thispagestyle{empty}
\begin{figure}[p]
    \centering
    \includegraphics[width=\textwidth]{figures/cara-gaussian-sigma=50.0/pp_stacked.pdf}
    \captionsetup{font=footnotesize} % Makes the note footnote-sized
    \caption{Optimal contracts with Gaussian distribution and CARA utility.}
    \label{fig:gaussian-cara-pp}
    \caption*{\textit{Note:} Top panel: optimal wage function $w(y)$. Bottom panel: agent's expected utility $U(v^*, a)$ Colors represent reservation utility. Dashed lines indicate that the first-order approach is invalid at that reservation utility. Bottom panel dots indicate the recommended action. The thin horizontal line indicates indifference between local maxima. arameters are as in figure \ref{fig:gaussian-log-pp}, but with CARA coefficient set to relative risk aversion of $1$ at the initial wealth.}
\end{figure}

Indeed, Theorem \ref{thm:main} shows that the first-order approach is valid for sufficiently high reservation utility. The examples illustrate the theorem idea in simple terms. For very low reservation utility, the first-order approach fails. The reason is that principal offers tough contracts, with zero pay for low output. Optimal contracts have a kink, and only start paying a positive wage relatively close to the intended action. Thus, the agent has a global deviation to basically give up and exert no effort. Thus, the first-order approach fails with very low reservation utility.

However, as \emph{reservation utility increases, the first-order approach becomes valid}. The reason is that the principal is forced to offer contracts with higher utility at the recommended action. This constraint moves the rightmost local maximum up. The rightmost maximum goes up faster than the leftmost local maximum. Thus, the rightmost maximum becomes the unique global maximum. In the examples, the first-order approach is valid for any positive reservation wage. The proof of Theorem \ref{thm:main} works with a somewhat different logic. The key observation is that the kink in the relaxed optimal contract moves to the left. This then implies that the agent's problem becomes more concave, and the first-order approach becomes valid.

Theorem \ref{thm:main} does not contradict existing results in the literature. The low reservation utility confirms the literature consensus that it is hard to guarantee that the first-order approach is valid is correct -- \emph{if we insist that the first-order approach is valid for all reservation utilities}. We see that, even in the Gaussian-log utility case, the first-order approach fails. Thus, any condition guaranteeing the first-order approach for all reservation utilities rules out basic examples and is therefore quite restrictive. The examples formally demonstrate the point made by \textcite{chaigneau2022should} and \textcite{conlon2009two}, that requiring the first-order approach to be valid for all reservation utilities excludes many interesting cases.

As a corollary to Theorem \ref{thm:main}, the problem is often tractable because we can use standard first-order approach arguments dating back to \textcite{holmstrom1978incentives}. We show that an optimal contract exists, is unique, and optimal wages have a simple formula:
\begin{equation}
    \label{eq:optimal-wage}
    w(y) = k \circ g\biggl(\mu + \lambda S(y | a_0)\biggr) \text{.}
\end{equation}

The function $k \circ g$ is determined by the utility function and the limited liability constraint. The score function $S(y | a_0)$ is determined by the distribution of output. The optimal action is $a_0$, and $\lambda$ and $\mu$ are Lagrange multipliers. We give closed-form solutions for these functions for a wide range of examples, and use these formulas to develop an algorithm to compute optimal contracts.

Another corollary is that optimal contracts are piecewise linear in a variety of examples. With log utility, $k \circ g$ is piecewise linear. It equals zero up to a point, and then increases linearly. And output distribution in an exponential family with linear sufficient statistic have linear score functions. Thus, with both of these conditions, the optimal contract is piecewise linear. Figures \ref{fig:gaussian-log-pp} and \ref{fig:poisson-log-pp} illustrate this in the Gaussian and Poisson cases. Note that the optimal contract is only piecewise linear when the first-order approach is valid. This is clear in Figure \ref{fig:gaussian-log-pp}, where the optimal contract is noticeably non-linear for reservation wage of $-\$1,000$, but otherwise piecewise linear. The CARA case in Figure \ref{fig:gaussian-cara-pp} illustrates that, without log utility, the optimal contract is not piecewise linear.

