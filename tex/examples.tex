\subsection{Examples Illustrating the Main Result}
Consider the following \textbf{Gaussian-log utility example}. A risk-neutral principal hires an agent to work. The agent chooses action $a$ at a cost $c(a)$ proportional to $a^2$. Output $y$, which accrues to the principal, is normally distributed with mean $a$ and standard deviation $\sigma$. The agent has some independent assets and log utility, so that her utility from a payment of $x \geq 0$ is $u(x) = \log(w_0 + x)$. For concreteness, suppose the principal is the owner of a small company and the agent is a professional manager. Illustrative values are $a$ choices in the ballpark of \$100{,}000, standard deviation $\sigma$ of \$20{,}000, and $w_0$ equal to \$50{,}000. The agent has reservation utility $\bar U$.

The principal designs a compensation contract with wage $w(y)$ to maximize profits. The optimal contract balances the goals of inducing effort, reducing the agent's risk, and providing the agent with her reservation utility. It is known that, without limited liability, an optimal contract does not exist. \textcite{mirrlees1999} showed that it is possible for the principal to get arbitrarily close to her perfect-information payoff using contracts that impose harsh punishments with small probability. Thus, we consider the case of limited liability, requiring $w(y) \geq 0$.\footnote{%
\textcite{moroni2014existence} show that, without limited liability, existence often fails if the agent has negative infinite utility from a finite wage.  Limited liability is a common and minimal way to avoid these types of existence issues.}

Figure \ref{fig:norm_log_opt} plots optimal contracts for a range of reservation utilities $\bar U$, with certainty equivalents from \$0 to \$40,000. The left panels plot the optimal wage schedules $w(y)$. The optimal contracts are piecewise linear option contracts. They have a flat region to the left where, due to limited liability, the agent is paid $w(y) = 0$, and an increasing linear region to the right, which gives the agent incentives to work. As reservation utility increases, contracts become more generous. Wage schedules shift upward, and the minimum output required to receive positive pay moves to the left.

% Gaussian - log utility figure
\begin{figure}[htb]
    \centering
    \begin{subfigure}[b]{0.45\textwidth}
        \includegraphics[width=\textwidth]{figures/norm_log_con_opt_a.pdf}
        \caption{Optimal contract}
        \label{fig:norm_log_con_opt}
    \end{subfigure}
    \hfill
    \begin{subfigure}[b]{0.45\textwidth}
        \includegraphics[width=\textwidth]{figures/norm_log_util_opt_a.pdf}
        \caption{Expected utility as a function of action}
        \label{fig:norm_log_util_opt}
    \end{subfigure}
    \vspace{0.5em}
    \caption{Optimal contracts in the Gaussian-log utility example}
    \label{fig:norm_log_opt}
    \captionsetup{font=footnotesize} % Makes the note footnote-sized
    \caption*{\textit{Note:} Panel a displays the wage function that solves the principal's profit maximization problem at different values of $\bar U$, and Panel b displays the agent's expected utility from a grid of possible actions given the corresponding wage functions from Panel a. The red line in Panel b plots the optimal action against optimal utility for the agent. The values of $\bar U$ are expressed in terms of certainty equivalents. The agent's utility is $u(x) = log(\frac{x}{20K})$, the cost function is $c(a) = \theta a^2$, where $\theta = .01$, and the agent's starting wealth is $w_0 = 50K$. The output is $y \sim N(a,(20K^2))$, where $a$ is the agent's action. Dotted lines indicate that the FOA is invalid. Our optimization routine has only one global IC constraint at $a=1e-3$. We conservatively say that the FOA is invalid if the Lagrange multiplier on the global IC constraint is greater than $1e-5$ or if deviating to an action between $0$ and $.1$ reduces the agent's utility by less than $1e-2$. }
\end{figure}

The right panel is crucial for understanding when the first-order approach is valid. It plots the agent's expected utility $U(v^*, a)$ as a function of $a$, given the optimal contract. The first-order approach fails whenever there is a binding non-local deviation. In these cases, solving the problem with only the local incentive constraint leads to incorrect solutions, as pointed out by \textcite{mirrlees1999}. Cases in which the first-order approach fails are denoted with dashed lines.

The first-order approach fails at the lowest reservation utility, with certainty equivalent of \$0. The optimal contract is a tough bargain. The agent is expected to exert effort $a$ equal to \$$177.2K$. The contract $w(y)$ specifies a payment of zero for any output lower than \$$113.1K$. Thus, the agent needs to produce relatively high output to get paid at all. This contract induces two local maxima in the function $U(v^*, a)$. The agent is indifferent between essentially giving up with a local maximum at $a \approx 0$ and a local maximum at the recommended effort level $a = \$177.2K$. This is consistent with the standard view in the literature. Option-like contracts are natural, and thus it is difficult to find general sufficient conditions under which the first-order approach holds. In particular, it is impossible to find sufficient conditions for the first-order approach that include the Gaussian-log utility example, and that work regardless of reservation utility.

Surprisingly, the first-order approach becomes valid as soon as reservation utility is slightly higher. The constraint of giving the agent higher utility forces the principal to offer contracts with a higher value of $U(v^*, a)$ at the recommended action. This constraint moves the rightmost local maximum of $U(v^*, a)$ up. The rightmost maximum goes up faster than the leftmost local maximum. Thus, the rightmost maximum becomes the unique global maximum. Numerically, the first-order approach is valid for any reservation utility with certainty equivalent above \$$600$.

Theorem \ref{thm:main} shows that this example is representative of a more general phenomenon. The theorem shows that, under certain conditions, the first-order approach is always valid for sufficiently high reservation utility. The most substantial condition is that greater output is evidence of greater effort (monotone likelihood ratio). Figures \ref{fig:grid_contract_opt} and \ref{fig:grid_util_opt} illustrate the generality of this point by plotting examples with different risk preferences and output distributions. All the examples display the same qualitative pattern, consistent with Theorem \ref{thm:main}. We stress that the monotone likelihood ratio assumption is crucial. For example, if output has a fat-tailed Student-$t$ distribution, there are examples where the first-order approach fails even for high reservation utilities. The reason is that positive and negative outliers are not very informative about effort, so the agent is not penalized for very poor outcomes, which generates global deviations to low effort.

% Example grid of optimal contracts
\begin{figure}[htbp]
    \centering
    \includegraphics[width=0.8\textwidth]{figures/grid_con_opt_a.pdf}
    \caption{Optimal contracts under different distributions and preferences}
    \label{fig:grid_contract_opt}
    \captionsetup{font=footnotesize} % Makes the note footnote-sized
    \caption*{\textit{Note:} This figure displays the wage function that solves the principal's profit maximization problem at different values of $\bar U$ for a grid of specifications. The log utility function is $u(x) = log(\frac{x}{20K})$. The CRRA utility function is $u(x) = \frac{1}{1-\gamma} \cdot \frac{x}{20K}^{1-\gamma}$, and $\gamma=2$. The CARA utility function is $\frac{1}{\alpha} \cdot -\exp(-\alpha \frac{x}{20K})$, and $\alpha=\frac{2}{5}$. The normal distribution is $y \sim N(a,(20K^2))$. The exponential distribution's rate parameter is $\frac{1}{a}$. The gamma distribution has shape parameter $n=2$, and scale parameter $a$. Formulas for all utility functions and output distributions are in the online appendix.  The cost function is always $c(a) = \theta a^2$, but $\theta = .01$ in the log utility examples, $\theta = .004$ in the CRRA examples and $\theta = .008$ in the CARA examples. Our optimization routine has only one global IC constraint at $a=1e-3$. We conservatively say that the FOA is invalid if the Lagrange multiplier on the global IC constraint is greater than $1e-5$ or if deviating to an action between $0$ and $.1$ reduces the agent's utility by less than $1e-2$.}
\end{figure}

% Example grid of agent utilities
\begin{figure}[htbp]
    \centering
    \includegraphics[width=0.8\textwidth]{figures/grid_util_opt_a.pdf}
    \caption{Expected utility as a function of action under different distributions and preferences}
    \label{fig:grid_util_opt}
    \captionsetup{font=footnotesize} % Makes the note footnote-sized
    \caption*{\textit{Note:} This figure displays the agent's expected utility from a grid of possible actions given the corresponding wage functions from figure \ref{fig:grid_contract_opt}. The specifications are the same as in figure \ref{fig:grid_contract_opt}.}
\end{figure}

Theorem \ref{thm:main} also guarantees existence and uniqueness of an optimal contract. Moreover, the solution has a simple calculus-based formula, as in the classic first-order approach literature following \textcite{holmstrom1978incentives}. The theorem shows that the optimal wage equals%
\footnote{%
These additional results are simple given the validity of the first-order approach. Our analysis and results are close to those in \textcite{jewitt2008moral}, albeit we cannot simply use their results due to slightly different assumptions. For example, they assume the \textcite{rogerson1985} conditions, which rule out examples such as the Gaussian distribution of output. Hence, we cannot simply use their results, although our proofs follow broadly similar lines.}
\begin{equation}
    \label{eq:optimal-wage}
    w(y) = k \circ g \left(\lambda + \mu S(y | a_0) \right) \text{.}
\end{equation}

The function $k \circ g$ is determined by the utility function and the limited liability constraint. $S(y | a_0)$ is the score function (also known as the likelihood ratio), determined by the distribution of output. The optimal action is $a_0$, and $\lambda$ and $\mu$ are Lagrange multipliers. In the case of log utility, $k \circ g$ is piecewise linear. For many output distributions, including the Gaussian, the score is linear. Hence, in the Gaussian example, the optimal contract is the piecewise-linear option contract from figure \ref{fig:norm_log_con_opt}.

Theorem \ref{thm:main} makes our limited liability setting particularly amenable to applications. As long as one considers a competitive setting with sufficiently high reservation utility, the main difficulties are resolved. Optimal contracts exist, are unique, and can be calculated trivially from equation (\ref{eq:optimal-wage}).

Section \ref{sec:applications} provides results to help researchers use the limited liability model. We provide simple calculus formulae for $k \circ g$ and $S$ covering a wide set of examples. These can be readily applied in both theoretical and numerical models. We provide accompanying code for numerical solutions.

An immediate corollary of equation (\ref{eq:optimal-wage}) is that piecewise linear contracts are optimal in a number of examples. Piecewise linear contracts are options, with a constant value up to a point, and increasing thereafter. The reason is the following. The function $k \circ g$ is piecewise linear when utility is $\log$, and the score function is linear for many common statistical distributions, including Gaussian, exponential, Poisson, and Gamma. More generally, any distribution in the exponential family with linear sufficient statistic has a linear score function. This can be seen in figures \ref{fig:norm_log_opt} and \ref{fig:grid_contract_opt}. Optimal contracts are slightly nonlinear for very low reservation utilities, but piecewise linear as soon as the first-order approach starts holding. This extends previous results that rationalize option-like contracts \textcite{innes1990limited, jewitt2008moral, chaigneau2024theory} and contributes to the literature rationalizing linear contracts \textcite{holmstrom1987aggregation,carroll2015robustness}.

\subsection{Relationship to the Literature}
At first glance, our results appear to contradict the standard view in the literature that the first order approach requires stringent conditions. In fact, the two are perfectly consistent. Consider the Gaussian example in figure \ref{fig:norm_log_opt}. When reservation utility is low, the optimal contract produces two distinct local maxima of the agent's utility, violating the first-order approach because the agent then has a profitable non-local deviation. This is exactly the kind of global incentive problem highlighted by \textcite{mirrlees1999}.  Likewise, the first-order approach fails in all the examples in figure \ref{fig:grid_contract_opt}. Hence, if one insists that the first-order approach must hold \emph{for every reservation utility}, many interesting examples must be ruled out. This is the standard view in the literature. Indeed, our examples imply that it is impossible to find sufficient conditions for the first-order approach that include basic examples like Gaussian-log utility, and that work regardless of reservation utility.

Our contribution is to show that \emph{if the reservation utility is sufficiently high}, those very same examples do satisfy the first-order approach.  The core idea is that requiring the principal to give the agent greater overall utility increases the payoff at the principal's intended action, pushing the local maximum at the intended action above other local maxima.

Historically, there are three generations of contributions to the first-order approach. The first generation was started by \textcite{mirrlees1999} in the mid 1970s, who showed that the first-order approach does not hold generally. Prior to \textcite{mirrlees1999}, the first-order approach was simply assumed with little justification. Subsequent papers then became aware that they had to assume validity of the first-order approach \textcite{holmstrom1978incentives}, and developed results that do not depend on the first-order approach \textcite{grossman1983,araujo2001general}.

The seminal papers in the second generation are \textcite{rogerson1985} and \textcite{jewitt1988justifying}. They provided sufficient conditions for the first-order approach. Rogerson's conditions are well known to be strict, ruling out many natural distributions, including the Gaussian and all examples in figure \ref{fig:grid_contract_opt}. Jewitt's conditions are more general, and include many interesting cases. Jewitt's main assumption is implicit, in that he requires his conditions to hold for all output levels, in particular ruling out limited liability \textcite{chaigneau2022should}. Restrictions similar to limited liability are known to be important to guarantee existence \textcite{moroni2014existence,jewitt2008moral}. In many examples that satisfy the Jewitt conditions, an optimal contract does not exist. This includes the Gaussian - log utility example without limited liability. The second-generation papers have been influential, and much of the theoretical and applied work to this day assumes these conditions.

Our own results are close to \textcite{jewitt1988justifying}. We impose similar conditions on the part of the contract where limited liability does not bind, while allowing for limited liability. In contrast, \textcite{jewitt1988justifying} requires the conditions to hold globally, ruling out limited liability, and sometimes creating existence issues \textcite{jewitt2008moral,moroni2014existence}. The proof of Theorem \ref{thm:main} shows that, with high reservation utility, the limited liability constraint binds with low probability. We then show, in similar lines to \textcite{jewitt1988justifying}, that the agent's problem becomes concave.

Our result is applicable because, in many examples, the first-order approach starts holding for relatively modest reservation utility, even when there are multiple local maxima. This can be seen in figures \ref{fig:norm_log_opt} and \ref{fig:grid_util_opt}. The intuition is that, if reservation utility is very high, then $U(v^*,a)$ is concave in $a$. As reservation utility increases, the functions $U(v^*,a)$ have to unwind into a function with a single peak. And the recommended action becomes optimal well before the function becomes concave. In this sense, our result has a limitation similar to other asymptotic results, such as the central limit theorem, that only holds for large enough sample sizes.

Finally, there is a third generation of work extending Jewitt's single-peakedness approach to multidimensional actions and richer information structures \cite[e.g.][]{conlon2009two,jung2015information,chade2020no,chaigneau2022should,jung2024proxy}. We believe our methods can be adapted to such environments, but we do not pursue that here.