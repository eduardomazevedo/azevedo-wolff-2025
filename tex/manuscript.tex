\documentclass[12pt]{article}
\usepackage{custom}

% Data reading
\newcommand{\data}[1]{\input{../output/constants/#1.txt}\unskip}

% External references for any separately compiled online appendix.
\usepackage{xr}
\externaldocument{"si"}

\begin{document}
\title{Broad Validity of the First-Order Approach in Moral Hazard%
\thanks{We are especially grateful to Lucas Maestri for originating key ideas in this paper. We also thank Humberto Moreira and Juuso Toikka for their valuable comments.}
}

\author{Eduardo M. Azevedo% 
\thanks{Wharton: 3733 Spruce Street, Philadelphia, PA 19104: eazevedo@wharton.upenn.edu, \url{https://eduardomazevedo.github.io/.}}
\and Ilan Wolff% 
\thanks{Wharton: 3733 Spruce Street, Philadelphia, PA 19104. ijwolff@wharton.upenn.edu.}
}

\date{
First version: March 7, 2025 \\
This version: \today}

\maketitle

\begin{abstract}%
The first-order approach (FOA) is the main tool in the study of the pure moral hazard principal-agent problem. Although many existing results rely on the FOA, its validity has been established only under relatively restrictive assumptions. We contribute three main findings.

First, we demonstrate in a broad array of examples that the FOA frequently fails when the agent's reservation utility is low (such as in principal-optimal contracts). However, the FOA holds when the agent's reservation utility is at least moderately high (such as in competitive settings where agents receive high rents).

Second, our main theorem shows that the FOA is valid in a standard limited liability model when the agent's reservation utility is sufficiently high. The theorem also establishes existence and uniqueness of the optimal contract.

Third, we use the FOA to derive tractable optimal contracts across a broad array of settings. These contracts are both simple and intuitive. Under log utility, option contracts are optimal for numerous common output distributions (including Gaussian, exponential, binomial, Gamma, and Laplace).%
\end{abstract}

\newpage

\section{Introduction}
\label{sec:introduction}
One of the workhorse models in economics is the principal-agent problem with moral hazard. The principal hires the agent to take an action $a$ in $\mathbb R _ +$ that affects the distribution $f(y|a)$ of output $y$. The principal can only condition payments on realized output, and designs a contract $w(y) \geq 0$ to provide incentives to the agent. The agent chooses the action $a$ to maximize her utility from wages minus her cost of effort $c(a)$ and the principal chooses the contract to maximize expected profits.

The main solution method for this problem is the first-order approach, which assumes that only local deviations in $a$ are binding, and yields a simple formula for the optimal contract.%
\footnote{%
See \textcite{holmstrom1978incentives}, the excellent survey \textcite{georgiadis2022contracting}, and recent work in \textcite{conlon2009two,kadan2017existence,chade2020no,chaigneau2022should,castro2024disentangling}.
}
Because of tractability, a vast literature simply assumes that the first-order approach is valid, and many applied papers make restrictive assumptions to avoid the issue of non-local deviations.%
\footnote{%
\textcite{mirrlees1999} showed that the first-order approach is not always valid. Examples of papers assuming the first-order approach include \textcite{jewitt2008moral, moroni2014existence, chaigneau2022should,castro2024disentangling}. Virtually all early work, including the seminal papers by \textcite{holmstrom1978incentives, holmstrom1979moral}, and \textcite{zeckhauser1970medical} assumes it. \textcite{holmstrom1979moral} notes that 
``one has to assume that the agent's optimal choice of action is unique
for the optimal [...] This assumption seems very difficult to validate [...] and regrettably we have to leave the question about its validity open.'' In applied work, restrictive assumptions such as linear contracts or binary effort are often made to avoid this issue. \textcite{edmans2009multiplicative} is an example of an important recent paper using both binary effort and linear contracts.
}

Unfortunately, existing sufficient conditions for the first-order approach to be valid are restrictive. The seminal papers are \textcite{rogerson1985} and \textcite{jewitt1988justifying}, followed by an extensive literature.%
\footnote{Important generalizations include \textcite{sinclair1994first,conlon2009two,jung2015information,chade2020no}, and \textcite{jung2024proxy}. \textcite{chaigneau2022should, chaigneau2024theory} develop sufficient conditions with limited liability.}
\textcite{kadan2017existence} summarize the general view that ``conditions facilitating the first-order approach are typically quite demanding.'' The key issues are elegantly explained by \textcite{chaigneau2022should} and \textcite{conlon2009two}, who says ``Unfortunately, the Jewitt conditions are tied to the concavity, not only of the technology, but also of the payment schedule. Thus, there are interesting cases where the Jewitt conditions should fail, since the payment schedule is often not concave. For example, managers often receive stock options, face liquidity constraints.''% 
\footnote{\textcite{conlon2009two} continues: ``
I believe that it is natural to expect the first-order approach itself to fail in such cases, since the agent's overall objective function will tend to be nonconcave [...]. The fact that CDFC and CISP imply concavity of the agent's payoff, regardless of the curvature of [the payment], then suggests that the CDFC/CISP conditions are very restrictive, not that the first-order approach is widely applicable.''%
}

Our main result, Theorem \ref{thm:main}, shows that the first-order approach is broadly valid as
long as the agent’s reservation utility is sufficiently high. The first-order approach holds even when
contracts are option-like and the agent’s problem has multiple local maxima. Theorem \ref{thm:main} also implies that the optimal contract exists, is unique, and is characterized by the standard simple formula. The main substantial condition is that the score of the output distribution is increasing in output.

The basic idea is simple, and clearly illustrated in basic examples as in Figure \ref{fig:gaussian-log-pp}. When reservation utility is very low -- say negative reservation wages -- the first-order approach often fails. Optimal contracts sometimes pay zero unless effort is quite high. This easily leads agents to be indifferent between equilibrium effort and giving up to nearly zero effort. However, as reservation utility increases, payoffs close to the intended action become more attractive, and the first-order approach becomes valid. Formally, the proof of Theorem \ref{thm:main} is built around the kink in the relaxed optimal contract moving far to the left, and this making the agent's problem more concave.

% Gaussian - log utility figure
\begin{figure}[p]
    \centering
    \includegraphics[width=\textwidth]{figures/log-gaussian-sigma=50.0/pp_stacked.pdf}
    \captionsetup{font=footnotesize} % Makes the note footnote-sized
    \caption{Optimal contracts with Gaussian distribution and log utility.}
    \label{fig:gaussian-log-pp}
    \caption*{\textit{Note:} Top panel: optimal wage function $w(y)$. Bottom panel: agent's expected utility $U(v^*, a)$ given optimal contract and action $a$. Colors represent reservation utility. Dashed lines indicate that the first-order approach is invalid at that reservation utility. Dots indicate the recommended action. The thin horizontal line indicates indifference between local maxima. Output has gaussian distribution with mean $a$ and standard deviation $50$, initial wealth is $50$ (both in thousands of dollars), and the cost function is $c(a) = a^2 / 30000$.}
\end{figure}

Section \ref{sec:examples} explains our results through simple examples and reconciles our findings with the literature. Section \ref{sec:model} gives definitions. Section \ref{sec:main-result} states Theorem \ref{thm:main}. Section \ref{sec:main-result-proof} outlines the proof. Section XXX provides a broad range of closed-form solutions for optimal contracts. Section XXX shows that optimal contracts are piecewise linear option contracts for log utility and output distributions in an exponential family with linear sufficient statistic (this has been independently discovered by \textcite{opp2025moral}). Section XXX provides numerical methods for both the case where the FOA is valid and in the more general case where it is not. Section XXX gives counter-examples and discusses limitations of our results. Section XXX discusses the relationship between our results and the literature.

\section{Examples and Literature Review}
\label{sec:examples}
Consider the following \textbf{Gaussian-log utility example}. A risk-neutral principal hires an agent to work. The agent chooses action $a$ at a cost $c(a)$ proportional to $a^2$. Output $y$, which accrues to the principal, is normally distributed with mean $a$ and standard deviation $\sigma$. The agent has some independent assets and log utility, so that her utility from a payment of $x \geq 0$ is $u(x) = \log(w_0 + x)$. For concreteness, suppose the principal is the owner of a small company and the agent is a professional manager. We consider $a$ choices in the order of \$100{,}000, standard deviation $\sigma$ of \$50{,}000, and $w_0$ equal to \$50{,}000. The agent has reservation wage $\bar w$.

The principal designs a compensation contract with wage $w(y) \geq 0$ to maximize profits. The optimal contract balances the goals of inducing effort, reducing the agent's risk, and providing the agent with her reservation utility.

Figure \ref{fig:gaussian-log-pp} plots optimal contracts for a range of reservation wages $\bar w$, with certainty equivalents from \$0 to \$50,000. The top panel plots the optimal wage schedules $w(y)$. The bottom panel plots the agent's expected utility $U(v^*, a)$ of an optimal contract $v^*$ as a function of action $a$. The intended action is denoted by the red dot in the bottom panel.

The bottom panel shows when cases the first-order approach is valid. The first-order approach fails whenever there is a binding non-local deviation. In these cases, solving the problem with only the local incentive constraint leads to incorrect solutions. Cases in which the first-order approach fails are denoted with dashed lines.

We see that the first-order approach is valid for all weakly positive reservation wages, and only fails for sufficiently negative reservation wages. This is an illustration of our main point, that the first-order approach is broadly valid in most interesting cases. Moreover, this is representative of a more general phenomenon, and not simply this example. Figures \ref{fig:poisson-log-pp} and \ref{fig:gaussian-cara-pp} consider a different distribution (the discrete Poisson distribution) and a different utility function (CARA). The same pattern holds.

% Poisson - log utility figure
\thispagestyle{empty}
\begin{figure}[p]
    \centering
    \includegraphics[width=\textwidth]{figures/log-poisson/pp_stacked.pdf}
    \captionsetup{font=footnotesize} % Makes the note footnote-sized
    \caption{Optimal contracts with Poisson distribution and log utility.}
    \label{fig:poisson-log-pp}
    \caption*{\textit{Note:} Top panel: optimal wage function $w(y)$. Bottom panel: agent's expected utility $U(v^*, a)$ Colors represent reservation utility. Dashed lines indicate that the first-order approach is invalid at that reservation utility. Bottom panel dots indicate the recommended action. The thin horizontal line indicates indifference between local maxima. Output has Poisson distribution with mean $a$, initial wealth is $50$ (both in thousands of dollars), and the cost function is $c(a) = a^2 / 2800$.}
\end{figure}


% Gaussian - CARA utility figure
\thispagestyle{empty}
\begin{figure}[p]
    \centering
    \includegraphics[width=\textwidth]{figures/cara-gaussian-sigma=50.0/pp_stacked.pdf}
    \captionsetup{font=footnotesize} % Makes the note footnote-sized
    \caption{Optimal contracts with Gaussian distribution and CARA utility.}
    \label{fig:gaussian-cara-pp}
    \caption*{\textit{Note:} Top panel: optimal wage function $w(y)$. Bottom panel: agent's expected utility $U(v^*, a)$ Colors represent reservation utility. Dashed lines indicate that the first-order approach is invalid at that reservation utility. Bottom panel dots indicate the recommended action. The thin horizontal line indicates indifference between local maxima. arameters are as in figure \ref{fig:gaussian-log-pp}, but with CARA coefficient set to relative risk aversion of $1$ at the initial wealth.}
\end{figure}

Theorem \ref{thm:main} demonstrates that these many examples are no coincidence. The theorem formally demonstrates that the first-order approach is valid for sufficiently high reservation utility. The examples illustrate the idea in simple terms. For very low reservation utility, the first-order approach fails. The reason is that principal offers tough contracts, with zero pay for low output. Optimal contracts have a kink, and only start paying a positive wage relatively close to the intended action. Thus, the agent has a global deviation to basically give up and exert no effort. Thus, the first-order approach fails with very low reservation utility.

However, as \emph{reservation utility increases, the first-order approach becomes valid}. The reason is that the principal is forced to offer contracts with higher utility at the recommended action. This constraint moves the rightmost local maximum of $U(v^*, a)$ up. The rightmost maximum goes up faster than the leftmost local maximum. Thus, the rightmost maximum becomes the unique global maximum. In the examples, the first-order approach is valid for any positive reservation wage. The proof of Theorem \ref{thm:main} works with a somewhat different logic. The key observation is that the kink in the optimal contract moves to the left, and this makes the agent's problem more concave. This then implies that the agent's problem becomes more concave, and the first-order approach becomes valid.

Theorem \ref{thm:main} does not contradict existing results in the literature. The low reservation utility case shows that the literature consensus that it is hard to guarantee that the first-order approach is valid for any reservation utility is correct -- \emph{if we insist that the first-order approach is valid for all reservation utilities}. We see that, even in the Gaussian-log utility case, the first-order approach fails. Thus, any conditions guaranteeing the first-order approach are valid for all reservation utilities must be quite restrictive. The examples formally demonstrate the point made by \textcite{chaigneau2022should} and \textcite{conlon2009two}, that requiring the first-order approach to be valid for all reservation utilities excludes many interesting cases.




The first-order approach fails at the lowest reservation utility, with certainty equivalent of -\$1,000. The optimal contract is a tough bargain. The agent is expected to exert effort $a$ equal to about \$$130,000K$. The contract $w(y)$ specifies a payment of zero for any output lower than about \$$100,000$. Thus, the agent needs to produce relatively high output to get paid at all. This contract induces two local maxima in the function $U(v^*, a)$. The agent is indifferent between essentially giving up with a local maximum at $a \approx 0$ and a local maximum at the recommended effort level $a \approx \$130,000$. This is consistent with the standard view in the literature. Option-like contracts are natural, and thus it is difficult to find general sufficient conditions under which the first-order approach holds. In particular, it is impossible to find sufficient conditions for the first-order approach that include the Gaussian-log utility example, and that work regardless of reservation utility.

Surprisingly, the first-order approach becomes valid as soon as reservation utility is slightly higher. The constraint of giving the agent higher utility forces the principal to offer contracts with a higher value of $U(v^*, a)$ at the recommended action. This constraint moves the rightmost local maximum of $U(v^*, a)$ up. The rightmost maximum goes up faster than the leftmost local maximum. Thus, the rightmost maximum becomes the unique global maximum. Numerically, the first-order approach is valid for any positive reservation wage.

Theorem \ref{thm:main} shows that this example is representative of a more general phenomenon. The theorem shows that, under certain conditions, the first-order approach is always valid for sufficiently high reservation utility. The most substantial condition is that the score is increasing in output (monotone likelihood ratio). That is, greater output is evidence of greater effort. Figures \ref{fig:grid_contract_opt} and \ref{fig:grid_util_opt} illustrate the generality of this point by plotting examples with different risk preferences and output distributions. All the examples display the same qualitative pattern, consistent with Theorem \ref{thm:main}. The monotone score assumption is crucial. For example, if output has a fat-tailed Student-$t$ distribution, there are examples where the first-order approach fails even for high reservation utilities. The reason is that positive and negative outliers are not very informative about effort, so the agent is not penalized for very poor outcomes, which generates global deviations to low effort.


Theorem \ref{thm:main} also guarantees existence and uniqueness of an optimal contract. Moreover, the solution has a simple calculus-based formula, as in the classic first-order approach literature following \textcite{holmstrom1978incentives}. The theorem imlpies that, beause we can use the first-order approach, the optimal wage equals%
\footnote{%
These additional results are simple given the validity of the first-order approach. Our analysis and results are close to those in \textcite{jewitt2008moral}, albeit we cannot simply use their results due to slightly different assumptions. For example, they assume the \textcite{rogerson1985} conditions, which rule out examples such as the Gaussian distribution of output. Hence, we cannot simply use their results, although our proofs follow broadly similar lines.}
\begin{equation}
    \label{eq:optimal-wage}
    w(y) = k \circ g\biggl(\mu + \lambda S(y | a_0)\biggr) \text{.}
\end{equation}

The function $k \circ g$ is determined by the utility function and the limited liability constraint. $S(y | a_0)$ is the score function (also known as the likelihood ratio), determined by the distribution of output. The optimal action is $a_0$, and $\lambda$ and $\mu$ are Lagrange multipliers.

Theorem \ref{thm:main} makes our limited liability setting particularly amenable to applications. As long as one considers a competitive setting with sufficiently high reservation utility, the main difficulties are resolved. Optimal contracts exist, are unique, and can be calculated trivially from equation (\ref{eq:optimal-wage}).

Section \ref{sec:applications} provides results to help researchers use the limited liability model. We provide simple calculus formulae for $k \circ g$ and $S$ covering a wide set of examples. These can be readily applied in both theoretical and numerical models. We provide accompanying code for numerical solutions.

An immediate corollary of equation (\ref{eq:optimal-wage}) is that piecewise linear contracts are optimal in a number of examples. In the case of log utility, $k \circ g$ is piecewise linear. For many output distributions, including the Gaussian, the score is linear. This holds for any distribution in the exponential family with linear sufficient statistic. This includes the Gaussian, exponential, Poisson, and gamma distributions. This is illustrated in figures XXX and XXX, where all the contracts where the first-order approach is valid are piecewise linear (although the rare contracts where the first-order approach is not valid are not). This linearity result has been independently discovered by \textcite{opp2025moral}, who also established many results not included here, especially for the case of the exponential distribution.

Piecewise linear contracts are options, with a constant value up to a point, and increasing thereafter. Our finding extends previous results that rationalize option-like contracts \textcite{innes1990limited, jewitt2008moral, chaigneau2024theory} and contributes to the literature rationalizing linear contracts \textcite{holmstrom1987aggregation,carroll2015robustness}.

\section{Model}
\label{sec:model}
\subsection{Model}

A risk-neutral \textbf{principal} hires an \textbf{agent} with limited liability. The agent has utility $u(x)-c(a)$ of receiving a \textbf{payment} $x \in \mathbb{R}_{+}$ and taking \textbf{action} $a$ in $\mathcal A \subseteq \mathbb{R}_+$. Both $u$ and $c$ are strictly increasing. \textbf{Output} $y \in \mathbb{R}$ depends on the agent's action and is distributed according to the \textbf{output density} $f(y|a)$.

A \textbf{contract} is a function $v:\mathbb{R} \rightarrow u(\mathbb{R}^+)$, which specifies the agent's utility $v(y)$ from the payment as a function of output $y$. Contracts are defined in terms of utility to simplify notation. To define the wage payments, let the \textbf{compensation cost function} $k$ be the inverse of $u$. A \textbf{wage function} is a function $w:\mathbb{R} \rightarrow \mathbb{R}^+$. The wage function associated with contract $v$ is $w(y) = k(v(y))$. Let $\mathcal C$ be the \textbf{set of feasible} contracts.

The \textbf{agent's utility from a contract} \(v\) given action \(a\) is
\[
U(v, a) := \int v(y) f(y|a) \, dy - c(a) \text{.}
\]

The \textbf{expected wage} is
\[
W(v, a) := \int k(v(y)) f(y|a) \, dy \text{.}
\]

The \textbf{cost minimization problem} given an \textbf{intended action} $a_0 \in \mathcal{A}$ and \textbf{reservation utility} $\bar U$ is to choose a contract $v$ to minimize the expected wage subject to the individual rationality (\ref{IR}) and global incentive compatibility (\ref{GIC}) constraints. Individual rationality requires that the agent's expected utility is at least reservation utility, and global incentive compatibility requires that choosing $a_0$ is optimal for the agent. The cost minimization problem is to choose $v$ in $\mathcal C$ to
\begin{align}
    \text{minimize} \quad & W(v) \nonumber \\
    \text{subject to} \quad &  U(v, a_0) \geq \bar{U} \tag{IR} \label{IR}, \\
    & U(v, a_0) \geq U(v, \hat{a}) \quad \forall \hat{a} \in \mathcal{A} \tag{GIC} \label{GIC}.
\end{align}

The \textbf{relaxed cost minimization problem} replaces the global incentive compatibility constraint with the local incentive compatibility constraint (\ref{LIC_relaxed}):
\begin{align}
    \text{minimize} \quad & W(v) \nonumber \\
    \text{subject to} \quad & U(v, a_0) \geq \bar{U} \tag{IR} \label{IR_relaxed}, \\
    & \partial _a U(v, a_0) = 0 \tag{LIC} \label{LIC_relaxed}. 
\end{align}


\subsection{Assumptions and Notation}

We now state needed assumptions. This section can be skimmed on a first reading.

\begin{assumption}[Regularity of utility, cost and density]
\label{assump:utility_cost}
The set of feasible actions $\mathcal{A} \subseteq \mathbb{R}_+$ is a compact interval including $0$. The utility function $u$ is strictly concave and smooth, has $\lim_{x\rightarrow\infty}u(x)=\infty$ and $\lim_{x\rightarrow\infty}u'(x)=0$. The cost function $c$ is strictly convex and smooth, and $c'(0) = 0$. The output density $f(y | a)$ is smooth in the action, $a$, for all outputs, $y$.
\end{assumption}

We now make technical assumptions that allow us to use Leibniz's rule of differentiation under the integral sign. For this, we impose restrictions on $k$, $f$ and on the set of feasible contracts. At the same time, we require the set of feasible contracts to be rich enough so that relevant contracts are not ruled out by assumption.

\begin{assumption}[Regularity of feasible contracts]
\label{assump:leibniz}
\begin{enumerate}
    \item Every feasible contract is measurable.
    \item Given any $a_1$ in $\mathcal{A}$ and a feasible contract $v$, there exists an integrable function $\theta(y)$ and a neighborhood of $a_1$ such that, for all $a$ in this neighborhood,
    \[
        \left|v \partial^n_a f(y|a)\right| \leq \theta(y)
    \]
    for $n = 0, 1, 2$ and for all $y$ in the support of $f(\cdot | a)$.
    \item For all $a$ in $\mathcal{A}$ and canonical contract $v$ (see definition \ref{def:canonical-contract} below),
    \[
        \Pi(v, a) < \infty\text{.}
    \]
    \item The set of feasible contracts is convex and includes all canonical contracts.
\end{enumerate}
\end{assumption}

We now consider substantive restrictions.

\begin{assumption}[Assumptions on the distribution of output]
    \label{assump:regularity_S}
    The output density function $f(y|a)$ satisfies:
    \begin{enumerate}
        \item The support of $f(\cdot | a)$ is the real line.%
        \item $F_a(y|a) < 0$ for all $y$ and $a$.
        \item For all $a$, the score $\partial_a \log f(y|a)$ is strictly increasing in $y$ and its image is $\mathbb R$ (this is known in the literature as the monotone likelihood ratio property or MLRP). 
        \item There exists $y_0$ in $\mathbb R$ such that, for all $y \leq y_0$ and $a$,
        $$
        f_{aa}(y|a) > 0 \text{.}
        $$
    \end{enumerate}
\end{assumption}

\begin{assumption}[Concavity and limit of inverse marginal utility]
\label{assump:concavity_inverse_marginal_utility}
The function ${k'}^{-1}$ is strictly concave and
\[
\lim _ {z \to \infty} z \frac{d}{dz} {k'}^{-1}(z)
\]
is finite.
\end{assumption}

The function $k'^{-1}$ is the same as \citeauthor{jewitt1988justifying}'s (\citeyear{jewitt1988justifying}) $\omega(z)$. So our assumption that $k'^{-1}$ is strictly concave corresponds to his assumption (2.12). The key difference is that \cite{jewitt1988justifying} does not have a limited liability constraint and requires that the first-order formula for an optimal contract holds globally. This requirement often implies that there is no solution to the principal's problem. For example in the Gaussian-log utility example there is no solution under Jewitt's assumptions (see derivation in Section \ref{sec:main-result-proof}).

\section{Main Result}
\label{sec:main-result}
We can now state our main result. We say that the first-order approach is valid for $(a_0, \bar U)$ if any solution to the relaxed cost minimization problem is a solution to the cost minimization problem.

Our main result is to show that the first-order approach is valid for sufficiently high reservation utility:

\begin{theorem}
    \label{thm:main}[Validity of the first order approach with high reservation utility]
    Given a regular intended action $a_0 > 0$, the first-order approach is valid for $(a_0, \bar U)$ for any sufficiently high $\bar U$.
\end{theorem}

The theorem formalizes the main point discussed in the introduction and examples. That the first order approach is valid under reasonable assumptions, as long as reservation utility is high.

The theorem simplifies the moral hazard problem because the relaxed problem is considerably more tractable. Proposition \ref{prop:relaxed-optimal-contract} shows that the relaxed problem has an almost everywhere unique solution given by a tractable formula. Combining theorem \ref{thm:main} and proposition \ref{prop:relaxed-optimal-contract} we obtain the following.

\begin{corollary}
    \label{thm:main}[Solution of the moral hazard problem with high reservation utility]
    Given a regular intended action $a_0 > 0$, there exist reservation utilities $U^*$ in $\mathbb R$ and $\bar U _R$ in $\mathbb R \cup \{\infty\}$ such that:
    \begin{enumerate}
        \item For $\bar U \geq \bar U _R$, neither the relaxed cost minimization problem nor the cost minimization problem are feasible.
        \item For $U^* \leq \bar U < \bar U _ R$, the cost minimization problem has an almost everywhere unique solution given by proposition \ref{prop:relaxed-optimal-contract}.
    \end{enumerate}
\end{corollary} 

\section{Proof of the Main Result}
\label{sec:main-result-proof}
The proof of the main result follows from two key propositions. Proposition \ref{prop:relaxed-optimal-contract} characterizes the solution of the relaxed problem. Proposition \ref{prop:concave} shows that, for sufficiently high reservation utility, the agent's problem is concave. We now state these propositions and explain the key steps in the argument. Appendix \ref{sec:appendix-proofs} contains the proofs.

For the relaxed problem to be well-defined, the derivative $\partial _a U(v,a)$ must exist. Remark \ref{rem:leibniz} shows that this is true under our assumptions, and moreover, that we can calculate the derivative by differentiating under the integral sign. Henceforth, we will use differentiation under the integral sign without referencing Remark \ref{rem:leibniz}.

Throughout this section, fix a regular intended action $a_0$. To simplify notation, we omit the dependence on $a_0$ whenever it is clear, writing, for example, $U(v)$ instead of $U(v, a_0)$.

\subsection{Solution to the Relaxed Problem}

We first show that the relaxed cost minimization problem has an almost everywhere unique solution, and that this solution has a simple formula in terms of Lagrange multipliers. This section is a minor extension of standard results on the relaxed problem going back to \textcite{holmstrom1978incentives} and \textcite{jewitt2008moral}.

The relaxed cost minimization problem is convex. Define its Lagrangian as
\begin{equation}
    \label{eq:lagrangian}
    \mathcal{L}(v,\lambda,\mu):=W(v)+\lambda\left(\bar{U}-U(v)\right)+\mu(-U_{a}(v))\text{.}
\end{equation}

Heuristically differentiating this Lagrangian with respect to $v(y)$ and setting the derivative to zero gives
\[
    k'(v(y)) f(y | a) = \lambda f(y | a) + \mu f_{a}(y|a_{0})\text{.}
\]

Dividing by $f(y | a)$ gives
\begin{equation}
\label{eq:foc}
\tag{FOC}
    k'(v(y)) = \lambda + \mu \frac{f_{a}(y|a_{0})}{f(y|a_{0})}\text{.}
\end{equation}

Equation (\ref{eq:foc}) is the key step in the standard first-order approach literature. Appendix \ref{sec:appendix-proofs} formally analyzes the convex program, demonstrates existence and uniqueness of the solution, and characterizes the optimal contract based on equation (\ref{eq:foc}).

The solution is best described with the following notation. Define the {\bf optimal expected wage} $\omega(\bar U)$ as the value of the infimum in the relaxed cost minimization problem. Define the \textbf{link function} $g:\mathbb R \rightarrow \mathbb R$ as\footnote{%
The link function's input, $z$, is a marginal dollar cost of providing one util to the agent  (measured in units of $\frac{\text{\$}}{\text{util}} $). The link function evaluated at $z$, $g(z)$, returns the utility level where $z$ is the marginal cost of utility to the agent. Any $z$ below $\frac{1}{u'(0)}$, which is the cheapest possible marginal cost, returns $g(z) = u(0)$.
}
\[
    g(z):=k'^{-1}\left(\max\left\{ \frac{1}{u'(0)},z\right\} \right) \text{.}
\]

Define the \textbf{score} function\footnote{The score is known as the likelihood ratio in the economics literature. We favor the term score to be in line with the modern scientific literature.} as in the statistics literature:
\[
    S(y | a) := \frac{\partial}{\partial a} \log f(y|a)
    =
    \frac{f_a(y|a)}{f(y|a)}
    \text{.}
\]
 
\begin{definition}
    \label{def:canonical-contract}
    A canonical contract $V(y | \lambda, \mu)$ is defined for $\lambda$ and $\mu$ in $\mathbb R$ as
    \begin{equation}
    \label{eq:canonical-contract}
    V\left(y | \lambda, \mu \right)
    :=
    g \biggl(\lambda + \mu S(y|a_0)\biggr) \text{.}
    \end{equation}
\end{definition}

The following proposition shows that the relaxed problem has a solution and characterizes the solution and Pareto frontier.

\begin{proposition}
    \label{prop:relaxed-optimal-contract}
    [Solution of the Relaxed Cost-Minimization Problem]
    There exists $\bar U_L$ in $\mathbb R$ such that:
    \begin{enumerate}
        \item (existence, uniqueness, and characterization) The relaxed problem has an almost everywhere unique solution $v^*(y|\bar U)$. There exist Lagrange multipliers $\lambda^*(\bar U) \geq 0$ and $\mu^*(\bar U) > 0$ such that the solution is almost everywhere equal to the canonical contract
    \[
        \label{eq:relaxed-optimal-contract}
        v^*(y|\bar U) :=
        V\biggl(y \big| \lambda^*(\bar U), \mu^*(\bar U)\biggr)
        \text{.}
    \]
        \item (comparative statics)
        \begin{itemize}
            \item For $\bar U \leq \bar U_L$, we have $\lambda^*(\bar U) = 0$. The relaxed optimal contract $v^*(\cdot , \bar U)$ and optimal expected wage $\omega(\bar U)$ do not vary with $\bar U$ in this range.
            \item For $\bar U > \bar U_L$, $\lambda^*(\bar U)$ is strictly increasing and $\lim_{\bar U \rightarrow \infty} \lambda^*(\bar U) = \infty$. The optimal expected wage $\omega(\bar U)$ is strictly increasing and strictly convex.
        \end{itemize}
    \end{enumerate}
\end{proposition}

The proposition shows that the solution to the relaxed cost minimization problem is unique and given by a canonical contract. Moreover, the Pareto frontier of profits and agent utility is convex.

The Gaussian-log utility example illustrates proposition \ref{prop:relaxed-optimal-contract}. The score is $(y-a) / \sigma^2$, the compensation cost function is $k(v)=\exp (v) - w_0$, and the link function is $\log \max \{z, w_0\}$. This implies the wage function
\[
 k\left(v^*(y|\bar U)\right)
 =
 \left[\lambda^*(\bar U) + \mu^*(\bar U) \frac{y-a}{\sigma^2} - w_0 \right] ^ +\text{.}
\]
This is the piecewise linear solution that we saw numerically in figure \ref{fig:gaussian-log-pp}. The convex Pareto frontier is illustrated in figure \ref{fig:pf}.

% Pareto frontier figure
\begin{figure}[ht]
    \centering
    \includegraphics{figures/log-gaussian-sigma=50.0/pareto_frontier.pdf}
    \caption{Pareto frontier in the Gaussian-log utility example}
    \label{fig:pf}
    \captionsetup{font=footnotesize} % Makes the note footnote-sized
    \caption*{\textit{Note:} This figure displays the Pareto frontier of the cost minimization problem and the relaxed cost minimization problem in the Gaussian-log utility example. The first-order approach is valid for all reservation utilities where the two sets coincide. Parameters are as in figure \ref{fig:gaussian-log-pp}.}
\end{figure}

The formula in proposition \ref{prop:relaxed-optimal-contract} is simply solving the first-order condition (\ref{eq:foc}) accounting for limited liability. The intuition is the following: the principal would like to always pay the agent $g(\lambda ^* (\bar U))$, which is the utility level where the marginal cost of providing utility to the agent is $\lambda ^* (\bar U)$. However, incentive compatibility requires that payment depends on whether there is statistical evidence of high effort. The $\mu$ term in the optimal contract pays more if the score is positive and less if the score is negative. Finally, the maximum ensures that limited liability is respected.

\subsection{High Reservation Utility} 

We now demonstrate that, for sufficiently high reservation utility, the solution to the relaxed cost minimization problem also solves the original cost minimization problem. To do so, we show that the relaxed problem's solution satisfies the global incentive compatibility constraint. This is achieved by demonstrating the stronger result that the agent's utility $U(v^*,a)$ is concave in $a$ at the relaxed optimal contract.

\begin{proposition} 
    \label{prop:concave} 
    [Concavity of the Agent's Problem for High Reservation Utility]
    There exists $U^*$ in $\mathbb{R}$ such that, for all $\bar U \geq U^*$ and all $a \in \mathcal{A}$,  
    $$ U_{aa}(v^* (y | \bar{U}), a) \leq 0 .$$ 
\end{proposition}

Theorem \ref{thm:main} is a direct consequence of this fact:

\begin{proof}[Proof of Theorem \ref{thm:main}]
    Take $\bar U \geq U ^*$. If the relaxed cost minimization problem has a solution $v^*(y | \bar U)$, proposition \ref{prop:concave} implies that the agent's expected utility $U(v^*,a)$ is concave in $a$. Therefore, the global incentive compatibility constraint is satisfied, and the solution to the relaxed cost minimization problem is also a solution to the cost minimization problem.
\end{proof} 

Figure \ref{fig:gaussian-log-cm} illustrates the proposition in the Gaussian-log utility example. We use the same parameters as in figure \ref{fig:gaussian-log-pp}, but consider the cost minimization problem with intended action $a_0 = \$100,000$. We see the same pattern as in figure \ref{fig:gaussian-log-pp}. For low reservation utility, there are multiple local maxima, and the first-order approach is invalid. For high reservation utility, there is only one local maximum, and the first-order approach is valid.

% Gaussian - log utility cost minimization problem figure
\begin{figure}[p]
    \centering
    \includegraphics[width=\textwidth]{figures/log-gaussian-sigma=50.0/cm_stacked.pdf}
    \captionsetup{font=footnotesize} % Makes the note footnote-sized
    \caption{Optimal contracts with Gaussian distribution and log utility in cost minimization problem with intended action $a_0 = \$100,000$.}
    \label{fig:gaussian-log-cm}
    \caption*{\textit{Note:} Top panel: optimal wage function $w(y)$. Bottom panel: agent's expected utility $U(v^*, a)$ given optimal contract and action $a$. Colors represent reservation utility. Dashed lines indicate that the first-order approach is invalid at that reservation utility. Dots indicate the recommended action. The thin horizontal line indicates indifference between local maxima. Intended action is $a_0 = \$100,000$ and other parameters are as in figure \ref{fig:gaussian-log-pp}.}
\end{figure}

The intuition for why the agent's problem is concave for high reservation utility is as follows. The relaxed optimal contract has two regions. When the outcome $y$ is below a threshold, the limited liability constraint binds, and the contract specifies a constant wage of $0$. When $y$ is above the threshold, the agent's payment is dictated by equation (\ref{eq:foc}). This kink introduces convexity into the agent's problem. As reservation utility increases, the kink moves to the left, so that the agent is paid with high probability. In the example, $a_0$ is fixed at \$100,000. As the kink moves to the left, the probability that the agent is paid a non-zero wage conditional on $a_0$ approaches $1$. 

Appendix \ref{sec:appendix-proofs} formalizes this argument in the proof of proposition \ref{prop:concave}. The proof has two key steps. First, it is shown that the probability that the agent receives zero payment converges to zero. When $\bar U$ is large, the typical payment received by the agent is large. So, if the agent could have a high impact on the probability of receiving a non-zero payment, she would have incentives to work harder than the intended effort level. This would violate the first-order condition, so the probability of receiving zero payment must converge to zero (lemma \ref{lem:threshold-outcome}).

The second key step is to show that this implies that the agent's expected utility is concave in $a$. The intuition is that the kink is very far to the left. Moreover, most of the increase in utility going from a payment of zero to the typical payment happens close to the kink. The proof shows that, due to this, the curvature of expected utility is dominated by the curvature of the cost function (lemma \ref{lem:inf_lambda_second_deriv}).

\section{Applications and Extensions}
\label{sec:applications}
\subsection{Closed Form Solutions}
Theorem \ref{thm:main} implies that optimal contracts have simple functional forms in parametric settings. Here we describe the solutions for common parametrization and note their properties.

Optimal contracts depend on effort cost, risk preferences (which determine the link function $g$), and on the distribution of output (which determines the score $S$). Tables \ref{tab:utility_functions} and \ref{tab:dists} provide basic formulae for the link and score functions that make up optimal contracts $g(\mu + \lambda S(y | a_0))$.

\begin{table}[ht]
    \centering
    \caption{Utility Functions, Link Functions, and Wage Functions}
    \label{tab:utility_functions}
    \renewcommand{\arraystretch}{1.5} % Adjust row height for better spacing
    \setlength{\tabcolsep}{10pt} % Slightly reduced column spacing
    \begin{tabular*}{\textwidth}{@{\extracolsep{\fill}}lccc}
        \toprule
        & \multicolumn{1}{c}{Utility Function} & \multicolumn{1}{c}{Link Function} & \multicolumn{1}{c}{Wage Function} \\ 
        & \( u(x) \) & \( g(z) \) & \( w(z) \) \\ 
        \midrule
        Log  & \multicolumn{1}{l}{\( \log(x + w_0) \)} & \multicolumn{1}{l}{\( \log(\max(w_0, z)) \)} & \multicolumn{1}{l}{\( (z - w_0)^+ \)} \\ 
        CRRA & \multicolumn{1}{l}{\( \frac{(x + w_0)^{1-\gamma}}{1-\gamma} \)} & \multicolumn{1}{l}{\( \frac{\max(w_0^\gamma, z)^{\frac{1-\gamma}{\gamma}}}{1-\gamma} \)} & \multicolumn{1}{l}{\( \left( (z^+)^{\frac{1}{\gamma}} - w_0 \right)^+ \)} \\ 
        CARA & \multicolumn{1}{l}{\( \frac{-\exp(-\alpha (x + w_0))}{\alpha} \)} & \multicolumn{1}{l}{\( -\frac{1}{\alpha \max(\exp(\alpha w_0), z)} \)} & \multicolumn{1}{l}{\( \frac{(\log^+ z - \alpha w_0)^+}{\alpha} \)} \\ 
        \bottomrule
    \end{tabular*}
    \captionsetup{font=footnotesize} % Match figure's note formatting
    \caption*{\textit{Note:} The utility function is the agent's utility from consumption given starting wealth $w_0$ and a transfer, $x$. The link and wage functions are in terms of $z$, which is a function of the outcome, $y$: \( z(y) = \lambda + \mu S(y|a_0) \).}
\end{table}
 

\begin{table}[ht]
    \centering
    \caption{Error Distributions}
    \label{tab:dists}
    \renewcommand{\arraystretch}{1.5} % Reduce vertical spacing
    \small % Reduce font size further
    \setlength{\tabcolsep}{6pt} % Reduce column spacing for compactness
    \resizebox{\textwidth}{!}{ % Ensures table fits within page width
    \begin{tabular}{l p{5.5cm} p{3.8cm} p{3.8cm}} % Keep all columns aligned and tight
        \toprule
        \textbf{Distribution} & \textbf{Probability Density} & \textbf{Score Function \( S(y|a) \)} & \textbf{Mean} \\ 
        \midrule
        Gaussian
        & \( \mathcal{N}(a, \sigma^2) \)
        & \( \frac{y - a}{\sigma^2} \) 
        & \( a \) 
        \\ 

        Log Normal
        & \( \frac{1}{y \sqrt{2\pi\sigma^2}} \exp \!\Bigl( -\frac{(\log(y)-a)^2}{2\sigma^2} \Bigr) \) 
        & \( \frac{\log(y) - a}{\sigma^2} \) 
        & \( \exp\!\Bigl(a + \frac{\sigma^2}{2}\Bigr) \) 
        \\ 

        Poisson
        & \( \frac{a^y e^{-a}}{y!} \) 
        & \( \frac{y - a}{a} \) 
        & \( a \) 
        \\ 

        Exponential 
        & \( \frac{1}{a} e^{-\tfrac{y}{a}} \) 
        & \( \frac{y-a}{a^2} \) 
        & \( a \) 
        \\ 

        Bernoulli
        & \( a^y (1 - a)^{1 - y} \), \( y \in \{0, 1\} \) 
        & \( \frac{y-a}{a-a^2} \)
        & \( a \) 
        \\  

        Geometric
        & \( \Bigl(1 - \frac{1}{a}\Bigr)^{y - 1} \Bigl(\frac{1}{a}\Bigr) \), \( y \in \{1,2,\dots\} \) 
        & \( \frac{y - a}{a^2 - a} \) 
        & \( a \) 
        \\ 

        Binomial
        & \( \binom{n}{y} a^y (1 - a)^{n - y} \), \( y \in \{0,\dots,n\} \) 
        & \( \frac{y - na}{a - a^2} \) 
        & \( n a \) 
        \\ 

        Gamma
        & \( f(y \mid n, a) = \frac{y^{n - 1} e^{-\tfrac{y}{a}}}{\Gamma(n)\, a^n} \) 
        & \( \frac{y - n a}{a^2} \) 
        & \( n a \)
        \\ 

        Student's \(t\)
        & \( 
        \frac{\Gamma\!\bigl(\tfrac{\nu + 1}{2}\bigr)}{\Gamma\!\bigl(\tfrac{\nu}{2}\bigr)\,\sqrt{\pi\nu}\,\sigma}
        \left(1 + \frac{1}{\nu} \,\frac{(y - a)^2}{\sigma^2}\right)^{-\tfrac{\nu + 1}{2}} 
        \) 
        & \( \frac{(\nu+1)(y-a)}{\nu\,\sigma^2 + (y-a)^2} \) 
        & \( a \) 
        \\ 

        Exponential Family
        & \( h(y)\,\exp\Bigl(\eta(a)\,T(y) - A(a)\Bigr) \) 
        & \( T(y)\,\frac{d\eta(a)}{da} \;-\; \frac{dA(a)}{da} \) 
        & \textit{(Not specified)} 
        \\ 

        \( y = a + X, X \sim h \)
        & \( g(y - a) \) 
        & \( -\frac{g'(y - a)}{h(y - a)} \) 
        & \( a + \mathbb{E}[X] \) 
        \\ 

        \( y = a X, X \sim h \) 
        & \( \bigl|\tfrac{1}{a}\bigr|\;h\!\Bigl(\tfrac{y}{a}\Bigr) \) 
        & \( -\frac{1}{a} - \frac{y}{a^2} \frac{g'(\frac{y}{a})}{g(\frac{y}{a})} \) 
        & \( a \,\mathbb{E}[X] \)
        \\ 
        \bottomrule
    \end{tabular}
    } % End resizebox
    \captionsetup{font=footnotesize} % Match figure's note formatting
    \caption*{\textit{Note:} This table presents probability the PDF, score function, and means of probability distributions as functions of the agent's chosen action, $a$.}
\end{table}



\textbf{Linearity}
Linear contracts play a prominent role in contract theory. It has long been noted that linear contracts are common, but that they only arise under relatively special assumptions \citep{holmstrom1987aggregation,carroll2015robustness}.

Our model yields piecewise linear contracts in a number of examples, as seen in the tables and in the Gaussian-log utility example. The key ingredients are log utility, which makes the wage function linear in the score, and a linear score function. This includes many distributions, because the score is linear for distributions in an exponential family with linear sufficient statistics. That is, when $f(y, a)$ is of the form
\begin{equation}
\label{eq:linear-exponential-family}
f(y | a)
=
\exp\left(\eta (a) y + A(a)\right) \\
\text{.}    
\end{equation}

We note this as follows:

\begin{remark}[Linear Contracts]
    Assume that utility is log (as in $u(x) = \log(w_0 + x)$) and that the distribution of outcomes is in an exponential family with linear sufficient statistic (as in equation \ref{eq:linear-exponential-family}). Then the relaxed optimal contract wage function is piecewise linear. This includes the Gaussian, exponential, Poisson, and gamma distributions.\footnote{Note that the remark covers distributions with support different than $\mathbb R$, such as the exponential. These distributions do not satisfy Assumption (\ref{assump:regularity_S}). Nevertheless, the proof of proposition \ref{prop:relaxed-optimal-contract} does not use that the support is $\mathbb R$, so the remark holds. However, our proof that the relaxed optimal contract is optimal does use that the support is $\mathbb R$. Therefore, we do not know whether Theorem 1 can be extended to different supports, and thus whether the remark can be extended to optimal contracts in the case of sufficiently high reservation utility. We conjecture based on numerical results that this is true and hope to include these facts in our next revision.}
\end{remark}

\textbf{Concave and Convex Contracts} The optimal contract is always at least partially convex because the limited liability constraint requires that $w(y) = 0$ for all $y$ less than some threshold, $\ubar y$. The region where the limited liability constraint does not bind, however, can be convex or concave depending on the agent's risk aversion.

\begin{remark}
    \label{rem:convexandconcave}
    Suppose the score function is linear, and let $\mu S(y | a_0) = k y$. Then, if the agent has CRRA utility, the region of the optimal contract where the limited liability constraint does not bind is convex if the risk aversion parameter $\gamma > 1$ and concave if $\gamma < 1$:\footnote{Assumption \ref{assump:concavity_inverse_marginal_utility} requires that $\gamma > \frac{1}{2}$.}  
    $$ w(y) = \left( \lambda + k y \right)^{\frac{1}{\gamma}} \text{ for } y > \ubar y. $$  

    If the agent has CARA utility, the same region of the optimal contract is logarithmic:
    $$ w(y) =  \frac{\log( \lambda + k y )}{\alpha} - w_0 \text{ for } y > \ubar y .$$
\end{remark}

The remark shows that the shape of the optimal contract depends heavily on the agent's utility function, and a wide variety of contracts are potentially compatible with the theory. Our result is contrary to \cite{conlon2009two}'s view that convex contracts, like CEO compensation with stock options, are incompatible with the first-order approach. In fact, the optimal contract with CRRA utility and $\gamma < 1$ closely approximates a CEO compensation package of stock options with many strike prices. To see this, consider a CEO who receives $n_{i}$ options with strike price $s_i$ that expire at the end of the year. The CEO's wage is 
$$ w(y) = \sum_i n_{i} (y - strike_i)^+, $$
where $y$ is the stock's end of year price price. 
The wage function's slope at $y$ is $\sum_{strike_i < y} n_{i}$; the wage is a piecewise-linear function where the slope increases as $y$ increases. The options package wage may be a discretization of the contract the model predicts for an agent with CRRA utility with $\gamma < 1$, and a linear score function. Its slope is also increasing on the outcome and equals $\frac{k}{\gamma}(\lambda + k y) ^\frac{1-\gamma}{\gamma}$ for some constant $k$. 

Remark \ref{rem:convexandconcave} affirms \citeauthor{holmstrom1987aggregation}'s (\citeyear{holmstrom1987aggregation}) view that the principle agent model does not robustly predict linear wages. The optimal wage is only linear if the agent's risk aversion parameter is precisely $\gamma=1$. However, if the agent's risk aversion is approximately one, the optimal contract will be approximately linear, and using a linear contract may be practical. 

\textbf{Limitations and numerical methods.}
The formulae have two main limitations. First, there is no general closed form solution for the Lagrange multipliers $\lambda$ and $\mu$. Thus, Theorem \ref{thm:main} guarantees that optimal contracts have the described functional form, but the Lagrange multipliers have to be calculated numerically. Second, as noted in Theorem \ref{thm:main}, these simple formulae do not hold for sufficiently low reservation utility.

To address these limitations, we provide accompanying code to numerically solve for optimal contracts. The numerical methods solve both the relaxed problem, and also the full cost minimization problem, in the case where only a finite number of global constraints bind. This is the case in all the experiments that we conducted. Both are convex optimization problems, and can be solved at trivial computational cost in all our experiments. The code implements all the common specifications in tables \ref{tab:utility_functions} and \ref{tab:dists}.


\section{Conclusion}
\label{sec:conclusion}
This paper shows that when an agent's reservation utility is sufficiently high, the longstanding concerns about the validity of the first-order approach largely disappear. Despite the possibility of multiple local maxima under low reservation utility, forcing the principal to provide higher overall utility eliminates global deviations, ensuring both existence of an optimal contract and applicability of the standard calculus-based approach. Our results hold under natural assumptions and extend to many commonly used distributions and preferences. This suggests that the first-order approach is especially relevant in competitive labor markets or other environments where agents have strong outside options.

In particular, our results show that the many existing results in the literature that have been proven assuming the first-order approach hold in this broader array of settings, include many settings with option-like contracts, and even contracts with wages that are convex functions of output. We hope that this result and the basic calculus formulae we provide will be useful in applied and theoretical work.

\bibliographystyle{aer}
\bibliography{bibliography}

\appendix
\newpage
\begin{center}
{\huge Appendix}{\huge\par}
\par\end{center}

\section{Proofs\label{sec:appendix-proofs}}
\subsection{Solution of the Relaxed Problem} 

Throughout this subsection, fix a reservation utility $\bar U$ and regular intended action $a_0$. The local IC constraint is only defined if $U(v, a)$ can be differentiated under the integral sign, so we begin by showing that differentiation under the integral at any action is possible for any feasible contract.

\begin{remark} 
    \label{rem:leibniz}
    Let $v$ be a feasible contract. Then, for all $a$ in $\mathbb{R}_{+}$, the first and second derivatives of $U(v,a)$ with respect to $a$ exist, are finite, and are given by
    \[
        U_{a}(v,a) = \int v(y) f_{a}(y|a)\,dy\text{,}
    \]
    \[
        U_{aa}(v,a) = \int v(y) f_{aa}(y|a)\,dy\text{.}
    \]
\end{remark} 

\begin{proof}
    The conditions for Leibniz's rule to hold are in Assumption \ref{assump:leibniz}.2.    
\end{proof} 

 We now derive some intermediate results to establish proposition \ref{prop:relaxed-optimal-contract}. Recall the definition of the Lagrangian from equation (\ref{eq:lagrangian}) and definition \ref{def:canonical-contract} of a canonical contract.

Note that canonical contracts uniquely minimize the Lagrangian:
\begin{lemma}
    \label{lem:lagrangian}
    Given $\lambda$ and $\mu$ in $\mathbb R$, there exists $v$ that minimizes the Lagrangian $\mathcal{L}(v, \lambda, \mu)$ among all feasible contracts. $v$ is $f(\cdot | a_0)$ almost everywhere equal to the canonical contract $V(y, \lambda, \mu)$.
\end{lemma}

\begin{proof}
The Lagrangian in equation (\ref{eq:lagrangian}) can be written as
\[
    \mathcal{L}(v, \lambda, \mu)
    =
    \int [k(v(y)) - \lambda v(y) - \mu v(y)S(y|a_0)]f(y|a_0)\, dy + \bar U
    \text{.}
\]

Differentiating the integrand with respect to $v(y)$ yields
\[
[k'(v(y)) - \lambda - \mu S(y|a_0)]f(y|a_0)
\]

and this is strictly convex in $v(y)$. Therefore, the integrand is minimized pointwise in $v(y)$ at $v(y) = V(y | \lambda, \mu)$. Hence, the infimum is attained, and any minimizer satisfies the desired formula $f(y | a_0)$ almost everywhere.

It only remains to show that $V(y | \lambda, \mu)$ is feasible. This follows from assumption \ref{assump:leibniz} Part 4.
\end{proof}

We now note that, given $\lambda$ there is a unique value of $\mu$ that solves the local IC constraint.

\begin{lemma}
    \label{lem:mu-tilde}
    Given $\lambda$ in $\mathbb R$, there exists a unique $\tilde \mu(\lambda)$ such that the canonical contract $V(y|\lambda, \tilde \mu(\lambda))$ satisfies the local IC constraint (\ref{LIC_relaxed}). Moreover, $\tilde \mu(\lambda) > 0$.
\end{lemma}

\begin{proof}
We have
\[
U_{a}(v)=\int v(y)f_{a}(y|a_{0})\,dy\text{.}
\]
If $\mu=0$, then $v(y)$ is constant, so
\[
    U_{a}(V(\cdot|\lambda,0,a_{0}))-c'(a_{0})<0\text{.}
\]

As $\mu\rightarrow\infty$, $V(y|\lambda,\mu,a_{0})$ converges pointwise to $u(\infty)$ if $S(y|a_0) > 0$ and to $u(0)$ if $S(y|a_0) < 0)$. Hence, because $a_0$ is regular, for large enough $\mu$,
\[
    U_{a}(V(\cdot|\lambda,\mu,a_{0}))-c'(a_{0})>0\text{.}
\]

Therefore, there exists at least one $\mu_{1}>0$ such that
\[
    U_{a}(V(\cdot|\lambda,\mu_{1},a_{0}))-c'(a_{0})=0\text{.}
\]

It only remains to prove that this solution $\mu_{1}$ is unique. To see this, note
that
\[
    U_{a}(V(\cdot|\lambda,\mu_{1},a_{0}))
\]
is weakly increasing in $\mu$. And, moreover, it is strictly increasing
at any solution $\mu_{1}$ because $c'(a_0) > 0$ implies that there is a positive measure of $y$ such that
$f_{a}(y|a_{0})>0$ and
\[
\lambda+\mu_{1}\frac{f_{a}(y|a_{0})}{f(y|a_{0})}>\frac{1}{u'(0)}\text{.}
    \]
\end{proof}

The lemma implies that the family of canonical contracts that satisfy (\ref{LIC_relaxed}) is a one-dimensional family indexed by $\lambda$. Define
\begin{equation}
    \label{eq:tildeV-definition}
    \tilde V (y | \lambda) := V(y | \lambda, \tilde \mu(\lambda))
\end{equation}

Define the \textbf{relaxed Pareto problem} as finding $v$ in $\mathcal C$ to
\begin{align}
    \text{minimize} \quad & W(v) - \lambda U(v) \nonumber \\
    \text{subject to} \quad & \partial _a U(v, a_0) = 0 \tag{LIC}
\end{align}

The next lemma shows that the contracts $\tilde V (y|\lambda)$ span the solutions to the Pareto problem:

\begin{lemma}
    \label{lem:pareto-problem}
    Given $\lambda$ in $\mathbb R$, the relaxed Pareto problem has a solution, and any solution is $f(y|a_0)$ almost everywhere equal to $\tilde V (y | \lambda)$.
\end{lemma}
\begin{proof}
For any $v$ satisfying (\ref{LIC_relaxed}),
\[
W(v) - \lambda U(v)
=
\mathcal{L}(v, \lambda, \tilde \mu(\lambda)) - \lambda \bar U
\text{.}
\]

Lemma \ref{lem:lagrangian} then implies that $W(v) - \lambda(v)$ is minimized over $\mathcal C$ by $\tilde V (y | \lambda)$, and that this solution is almost-everywhere unique. This contract satisfies (\ref{LIC_relaxed}) by the definition of $\tilde \mu(\lambda)$.
\end{proof}

Define the expected wage and utility attained by these contracts as
\[
\begin{array}{rcl}
\tilde U (\lambda) & := & U(\tilde V(\cdot | \lambda), a_0) \text{,} \\
\tilde W (\lambda) & := & W(\tilde V(\cdot | \lambda), a_0) \text{.}
\end{array}
\]

\begin{lemma}
    \label{u-tilde-increasing}
    $\tilde U (\lambda)$ is strictly increasing.
\end{lemma}

\begin{proof}
Consider $\lambda _1 < \lambda _2$ with optima $v_1 := \tilde V (\cdot | \lambda_1)$ and $v_2 := \tilde V (\cdot | \lambda_2)$. By optimality,
\[
W(v_1) - \lambda_1 U(v_1) \leq W(v_2) - \lambda_1 U(v_2)
\text{,}
\]
and
\[
W(v_2) - \lambda_2 U(v_2) \leq W(v_1) - \lambda_2 U(v_1)
\text{.}
\]

Adding the inequalities,
\[
(\lambda_2 - \lambda_1) \cdot \left(\tilde U (\lambda _2) - \tilde U (\lambda _ 1) \right) \geq 0
\text{.}
\]

Therefore, $\tilde U$ is non-decreasing. It only remains to show that $\tilde U$ is strictly increasing. To reach a contradiction, assume that $\tilde U(\lambda _ 2) = \tilde U (\lambda _ 1)$. Optimality implies that $\tilde W (\lambda _1) = \tilde W (\lambda _2)$.  Let $\bar \lambda = (\lambda_1 + \lambda_2) / 2$. We have $\tilde U (\bar \lambda) = \tilde U(\lambda _ 1)$. By our assumptions on the score, $v_1$ and $v_2$ differ in a set of positive measure. By strict convexity of $k$, it follows that $\tilde W (\bar \lambda) < \tilde W (\lambda _ 1)$. This contradicts the optimality of $v_1$.
\end{proof}

The proof of proposition \ref{prop:relaxed-optimal-contract} follows from collecting these results.

\begin{proof}[Proof of Proposition \ref{prop:relaxed-optimal-contract}]
Let $\bar U _ L : = \tilde U (0)$ and $\bar U _ R := \lim _ {\lambda \rightarrow \infty} \tilde U (\lambda)$.

\textbf{Part 1.}

The definition of $\bar U _ R$ implies that the relaxed problem is not feasible for $\bar{U} \geq \bar{U}_R$, as desired.

\textbf{Part 2.}

Let $\lambda ^*(\bar U)$ be $0$ if $\bar U \leq \bar U _L$ and be the inverse of $\tilde U$ for $\bar U _L < \bar U < \bar U _R$. Lemma \ref{u-tilde-increasing} implies that $\lambda^*$ is well defined and that $\lambda^*(\bar U) \geq 0$.

Let $\mu ^* (\bar U) := \tilde \mu (\lambda ^* (\bar U))$. Lemma \ref{lem:mu-tilde} implies that $\mu^*(\bar U) > 0$. Let $v^*(y | \bar U) := \tilde V (y | \lambda ^* (\bar U))$. Note that this coincides with the definition of $v^*$ in the proposition statement.

Lemma \ref{lem:pareto-problem} implies that $v^*(y | \bar U)$ solves the Pareto problem given $\lambda^*(\bar U)$. This implies that $v^*(y | \bar U)$ also solves the relaxed cost minimization problem given $\bar U$. Likewise, \ref{lem:pareto-problem} implies that the solution is unique almost everywhere.

\textbf{Part 3, nonbinding individual rationality case.}

This follows from the definition of $\lambda^*(\bar U)$ which equals $0$ in this range.

\textbf{Part 3, binding individual rationality case.}
$\lambda ^*$ strictly increasing follows from lemma $\ref{u-tilde-increasing}$ and the strict convexity of $\omega$ follows from $v^*(\cdot | \lambda)$ being different for each value of $\lambda$ and from $k$ being strictly convex.
    
\end{proof}

 \subsection{Proof of Proposition \ref{prop:concave}}
 \label{subsec:proof-derivs}  

 This section demonstrates that the second derivative of agent's utility is negative for sufficiently high reservation utility. The bulk of the section is spent demonstrating the same result for $\lambda$ sufficiently high, and the main result follows as an immediate corollary. Throughout this section, we use the $\tilde V (y | \lambda)$ notation defined in equation (\ref{eq:tildeV-definition}) because we are focused on the agent's problem as $\lambda$ becomes large. Although $\tilde \mu(\lambda)$ is a function of $\lambda$ (Lemma \ref{lem:mu-tilde}) we abuse notation by omitting dependence on $\lambda$ and writing $\tilde{\mu}$ to denote $\tilde{\mu}(\lambda)$. 

 The section proceeds as follows. Lemmas \ref{lem:v_derivs} and \ref{lem:second-derivative} derive a convenient formula for the second derivative of agent's utility with respect to effort. Lemma \ref{lem:threshold-outcome} shows that the agent's probability of receiving payment approaches 1 as $\lambda$ approaches infinity. Lemma \ref{lem:inf_lambda_second_deriv} shows that the second derivative of agent's utility is negative for $\lambda$ sufficiently high. We then use these lemmas to prove the main result.

We begin with an equation for the general form of agent's utility and its derivatives for an arbitrary contract. 
\begin{lemma} 
\label{lem:v_derivs}
    Given a contract, $v$, $U$ and its derivatives evaluated at $a$ equal
    \begin{align*}
    U(v,a) &= \int v \cdot f(v|a) \, dy - c(a), \\
    U_{a}(v,a) &= \int v \cdot S(y|a) f(y | a) \, dy - c'(a), \\
    U_{aa}(v,a) &= \int v \cdot \left( S^2(y|a) + S_{a}(y|a) \right) f(v|a) \, dy - c''(a).
    \end{align*}
\end{lemma} 

\begin{proof}
The expression for $U(v,a)$ is the definition. 
Differentiating $U(v,a)$ with respect to $a$ yields 
\[
U_a(v,a) = \int v f_a(v|a) \, dv - c'(a).
\]
The formula for $U_a$ follows from the fact that $f_a(v|a) = f(v|a) S(v|a)$.
Differentiating $U_a(v,a)$ with respect to $a$ gives
\begin{align*}
U_{aa}(v,a) &= \int v \frac{\partial}{\partial a} \left[ f(v|a) S(v|a) \right] dv - c'(a) \\
&= \int v \left[ f_a(v|a) S(v|a) + f(v|a) S_a(v|a) \right] dv - c''(a) \\
&= \int v \left[ f(v|a) S(v|a) S(v|a) + f(v|a) S_a(v|a) \right] dv - c''(a) \\
&= \int v \left[ S^2(v|a) + S_a(v|a) \right] f(v|a) \, dv - c''(a) .
\end{align*} 
\end{proof}

\noindent We now use Lemma \ref{lem:v_derivs} to derive the following equations for the agent's utility and its derivatives.  
\begin{lemma}
\label{lem:second-derivative}
    Given a canonical contract, $v(y) := \tilde V (y | \lambda)$, the agent's utility and its derivatives evaluated at $a$ are  
    \[
    \begin{aligned}
        U \left( v, a \right) 
        &= g(\lambda) + \tilde{\mu} \int \Delta g(y | \lambda) \cdot S(y|a_0) f(y | a) \, dy - c(a), \\
        U_a \left( v, a \right) 
        &= \tilde{\mu} \int \Delta g(y | \lambda) \cdot S(y|a) \cdot S(y|a_0) f(y | a) \, dy - c'(a), \\
        U_{aa}\left( v, a \right) 
        &= \tilde{\mu} \int \Delta g(y | \lambda) \left( S^2(y|a) + S_a(y|a) \right) \cdot S(y|a_0) f(y | a) \, dy - c''(a),
    \end{aligned}
    \]
    where 
    \[
    \Delta g(y | \lambda) = \frac{g(\lambda + \tilde{\mu} S(y |  a_0)) - g(\lambda)}{\tilde{\mu} S(y |  a_0)}.
    \]
\end{lemma}

\begin{proof}
    The utility function, \( U \), evaluated at the canonical contract, $
    \tilde{V}(y|\lambda)$, and action, $a$, is
    \[
    U \left( \tilde{V}(y|\lambda), a \right) = \int \tilde{V}(y|\lambda) f(y|a) \, dy.
    \]
    Recall that $\tilde{V}(y|\lambda) = g\left( \lambda + \tilde{\mu} S(y|a_0) \right)$. Substituting in yields 
    \[
    U \left( \tilde{V}(y|\lambda), a \right) = \int g\left( \lambda + \tilde{\mu} S(y|a_0) \right) f(y|a) \, dy.
    \]
    Adding and subtracting \( g(\lambda) \), we rewrite the integral:
    \[
     U \left( \tilde{V}(y|\lambda), a \right) = g(\lambda) + \int \left[ g\left( \lambda + \tilde{\mu} S(y|a_0) \right) - g(\lambda) \right] f(y|a) \, dy.
    \]
    Using the definition $ \Delta g(y | \lambda) = \frac{g(\lambda + \tilde{\mu} S(y |  a_0)) - g(\lambda)}{\tilde{\mu} S(y |  a_0)} $ and multiplying by $\frac{\tilde{\mu} S(y|a_0)}{\tilde{\mu} S(y|a_0)}$ yields 
     $$ U \left( \tilde{V}(y|\lambda), a \right) = g(\lambda) + \tilde{\mu} \int \Delta g(y | \lambda) \cdot S(y|a_0) f(y | a) \, dy - c(a) .$$
     The lemma follows from \ref{lem:v_derivs}'s formula for the derivatives applied to  $v(y) = \Delta g(y | \lambda) \cdot S(y|a_0)$. 
\end{proof} 

We now demonstrate that for $\lambda$ sufficiently large the agent receives a strictly positive payment for an arbitrarily large portion of the support of $f(y|a)$ for any $a > 0$. We begin by defining some additional notation. 

\begin{definition}
    \label{def:threshold-score}
     The \textbf{threshold score}, $\ubar{S}(\lambda)$, is the maximum score such that the agent receives no payment. It is the score that solves 
    \[
    \lambda + \tilde \mu \ubar{S}(\lambda) = \frac{1}{u'(0)} \text{,}
    \]
    or, equivalently,
    \[
    \ubar{S}(\lambda) := \frac{1}{\tilde{\mu} u'(0)} - \frac{\lambda}{\tilde{\mu}} \text{.}
    \]  
    The threshold score is well defined because $\tilde{\mu} > 0$ by Lemma \ref{lem:mu-tilde}.
\end{definition}

\begin{definition}
    \label{def:threshold-outcome}
    The \textbf{threshold outcome}, $\ubar{y}(\lambda)$, is the outcome which induces the threshold score, $\ubar{S}(\lambda)$:
    $$ \ubar{y}(\lambda) = S^{-1}(\ubar{S}(\lambda, \tilde{\mu}) |  a_0) .$$ 
    A score that satisfies the equation exists by Assumption \ref{assump:regularity_S}.3, which states that the score's image is $\mathbb{R}$. 
\end{definition} 
    
\begin{lemma} 
    \label{lem:threshold-outcome}
    The threshold outcome approaches negative infinity as $\lambda$ approaches infinity: 
    $$ \lim _ {\lambda \rightarrow \infty} \ubar{y}(\lambda) = -\infty . $$ 
\end{lemma}

\begin{proof}
    Lemma \ref{lem:second-derivative}'s result for $U_a$ evaluated at $a_0$ yields
    $$ U_a  \left( \tilde{V}(y|\lambda), a_0 \right) = \tilde{\mu} \int  \Delta g(y | \lambda) \cdot S(y|a_0)^2 f(y|a_0) - c'(a_0). $$
    The local incentive compatibility constraint requires that $U_a  \left( \tilde{V}(y|\lambda), a_0 \right)=0$. It follows that    
    $$  \tilde{\mu} \int  \Delta g(y | \lambda) \cdot S(y|a_0)^2 f(y|a_0) = c'(a_0). $$ 
    Observe that the term inside the expectation is weakly positive because $\Delta g(y | \lambda) \geq 0 $ by the monotonicity of $g$ and $S(y|a_0)^2 \geq 0$. 
    Therefore, 
    $$
    c'(a_0) \geq
    \tilde{\mu} \int _{S(y|a_0) \leq \ubar S(\lambda)} \Delta g(y | \lambda) \cdot S^2(y|a_0) f(y|a_0) \, dy \text{.}
    $$
    By definition, for all $y$ in the domain of integration, $g(y)=u(0)$. Substituting in to $\Delta g(y | \lambda)$ yields
    $$
    c'(a_0) \geq
    \tilde{\mu} \int _{S(y|a_0) \leq \ubar S(\lambda)} \frac{(u(0) - g(\lambda))}{\tilde{\mu} S(y|a_0)} \cdot S^2(y|a_0) f(y|a_0) \, dy \text{.}
    $$
    We simplify and use the fact that $f_a(y|a_0) = f(y|a_0) S(y|a_0)$ 
    $$ c'(a_0)
    \geq (u(0) - g(\lambda)) \cdot \int_{S(y|a_0) \leq \ubar S(\lambda)} f_a(y|a_0) \, dy .$$ 
    Integrating yields: 
    $$ c'(a_0)
    \geq (u(0) - g(\lambda)) \cdot F_a(\ubar y (\lambda) | a_0) .$$ 
    Observe that $u(0) - g(\lambda) \rightarrow -\infty$, and $ F_a(\ubar y (\lambda) | a_0) < 0 $ by Assumption \ref{assump:regularity_S}.2. Suppose $F_a(\ubar y (\lambda) | a_0)$ does not approach $0$. Then we have $c'(a_0) \geq \infty$. By contradiction, $F_a(\ubar y (\lambda) | a_0) \rightarrow 0$. 

    The threshold outcome, $\ubar y (\lambda)$, cannot converge to infinity because $\ubar S(\lambda)$ is negative for $\lambda$ sufficiently large because $S(\infty | a_0) = \infty$ by Assumption \ref{assump:regularity_S}.3, and $\infty > \ubar S(\lambda)$. If $\ubar S (\lambda)$ does not converge to negative infinity, then there is a subsequence that converges to a finite number. However, $F_a(c | a_0) < 0$ for any $c$ by Assumption \ref{assump:regularity_S}.2. The proposition is proven by contradiction. 
\end{proof}    

\begin{lemma}
    \label{lem:inf_lambda_second_deriv}
    As $\lambda$ approaches infinity, the limit of the supremum of the second derivative of agent's utility at any action $a > 0$  is strictly negative: 
    $$
    \limsup_{\lambda \to \infty}
    \,
    U_{aa}\left( \tilde{V}\left( y |  \lambda \right), a \right) < 0\text{.}
    $$
\end{lemma} 

\begin{proof}
By Lemma \ref{lem:second-derivative}, 
    $$ U_{aa}\left( \tilde{V} \left( y |  \lambda \right), a \right) = g(\lambda) + \tilde{\mu} \int \Delta g(y | \lambda) \cdot S(y|a_0) f(y | a) \, dy - c(a) $$ 
    Lemma \ref{lem:y_0} states that there exists a $y_0$ such that the integrand is negative for all $y < y_0$. It follows that 
    $$
    U_{aa}\left( \tilde{V} \left( y |  \lambda \right), a \right) \leq
    \tilde{\mu} \int_{y_0}^{\infty} 
    \Delta g(y | \lambda)
    \left( S^2(y|a) + S_a(y|a) \right) \cdot S(y|a_0)
    \cdot f(y | \hat a) \, dy - c''(a) \text{.}
    $$ 
    Let $Y_+$ be the set of $y \geq y_0$ where this integrand is positive. Then 
    $$
    U_{aa}\left( \tilde{V} \left( y |  \lambda \right), a \right) \leq
    \tilde{\mu} \int_{Y_+}
    \Delta g(y | \lambda)
    \left( S^2(y|a) + S_a(y|a) \right) \cdot S(y|a_0)
    \cdot f(y | \hat a) \, dy - c''(a) \text{.}
    $$
     Lemma \ref{lem:threshold-outcome} implies that for $\lambda$ sufficiently large, $\ubar y (\lambda, a_0) \leq y_0$. Recall that for $y>\ubar y$, $g(y) = k'^{-1}(\lambda + \mu S(y|a_0))$. The concavity of $k'^{-1}$ (Assumption \ref{assump:concavity_inverse_marginal_utility}) implies that $\Delta g$ is decreasing for $y$ in $Y_+$. Therefore,
    $$
    U_{aa}\left( \tilde{V} \left( y |  \lambda \right), a \right) \leq
    \tilde{\mu} 
    \Delta g(y_0 | \lambda)
    \int_{Y_+}
    \left( S^2(y|a) + S_a(y |a) \right) \cdot S(y_0|a_0)
    \cdot f(y | \hat a) \, dy - c''(a) \text{.}
    $$
    Because $g(y)$ is concave for $y > \ubar y$ and $y_0 > \ubar y$
    \begin{align*}
    \tilde{\mu} \Delta g(y_0 | \lambda)
    &\leq \tilde{\mu} g'(\lambda + \tilde{\mu} S(y|a_0)) \\
    &= \frac{\tilde{\mu}}{\lambda + \tilde{\mu} S(y|a_0)} \cdot (\lambda + \tilde{\mu} S(y|a_0)) g'(\lambda + \tilde{\mu} S(y|a_0)). 
    \end{align*}
    Lemma \ref{lem:mu_lambda} states that $\frac{\tilde{\mu}}{\lambda} \rightarrow 0$ as $\lambda \rightarrow \infty $, and Assumption \ref{assump:concavity_inverse_marginal_utility} implies that $(\lambda + \tilde{\mu} S(y_0|a_0)) g'(\lambda + \tilde{\mu} S(y_0|a_0))$ has a finite limit. It follows that 
    $$ U_{aa}\left( \tilde{V} \left( y |  \lambda \right), a \right) \leq - c''(a) < 0 $$ 
    with the last inequality implied by the strict convexity of the cost function (Assumption \ref{assump:utility_cost}). 
\end{proof} 

\noindent We now prove Proposition \ref{prop:concave}. 
\begin{proof}
    We first prove that there exists $\lambda_0$ such that, for all $\lambda \geq \lambda _0$ and $a$ in $\mathcal A$,
    \[
        U_{aa}(\tilde V(y | \lambda), a) \leq 0\text{.}
    \]
    
    To reach a contradiction, assume that this is not the case. Then there exists a sequence of $\lambda_n \rightarrow \infty$ and $a_k$ such that
    \[
        U_{aa}(\tilde V(y | \lambda_n), _n) > 0\text{.}
    \]
    
    Because $\mathcal A$ is compact, we can take a convergent subsequence where $a_k \rightarrow a_1$. Therefore,
    \[
        \limsup _ {\lambda \rightarrow \infty}
        U_{aa}(\tilde V(y | \lambda), a_1) > 0\text{.}
    \]
    This contradicts Lemma \ref{lem:inf_lambda_second_deriv}.
    
    For any $\lambda \geq \lambda_0$, we thus have that $U(\tilde V (y | \lambda), a)$ is concave in $a$. The proposition is proven by letting $U^*$ be the solution to $\lambda^*(U^*) = \lambda_0$. The agent's problem is concave in $a$ for $\bar{U} \in [U^*, U_{R}]$ because $\lambda^*(\bar{U})$ is monotonic by Proposition \ref{prop:relaxed-optimal-contract}.4 
\end{proof}

\noindent We conclude the section by proving two minor lemmas that we used in the above proofs. 

\begin{lemma}
    \label{lem:y_0} 
    There exists $y_0$ such that for all $y < y_0$, 
     $$ \Delta g(y | \lambda) \left( S^2(y|a) + S_a(y|a) \right) \cdot S(Y|a_0) \leq 0 .$$
\end{lemma} 

\begin{proof} 
    The lemma follows from these 3 facts. 
    \begin{enumerate}
        \item $\Delta g(y | \lambda) \geq 0$ 
        \item There exists $y_0$ such that  $S(Y|a_0) < 0$ for $y<y_0$
        \item There exists $y_0$ such that  $\left( S^2(y|a) + S_a(y|a) \right) > 0$ for $y \leq y_0$. 
    \end{enumerate} 
    The monotonicity of $g$ implies the first fact. The second fact follows from Assumption \ref{assump:regularity_S}.3. Assumption \ref{assump:regularity_S}.4 and the following algebra implies the final fact. 
    $$  S^2(y | \hat a) + S_a(y|a)
        = S^2(y | \hat a) + \frac{d}{da}\frac{f_a(y|a)}{f(y|a)}
        = S^2(y | \hat a) + \frac{f_{aa}(y|a)}{f(y|a)} - \frac{f_a(y|a)^2}{f(y|a)^2}
        = \frac{f_{aa}(y|a)\text{.}}{f(y|a)}. $$  
    The density is always positive, so the final expression is positive whenever $f_{aa}(y|a)$ is positive. 
\end{proof}

\begin{lemma}
    \label{lem:mu_lambda} 
    As $\lambda \rightarrow \infty$, $\frac{\tilde{\mu}}{\lambda} \rightarrow 0$. 
\end{lemma} 

\begin{proof}
    Lemma \ref{lem:threshold-outcome} and assumption \ref{assump:regularity_S}.3 imply that $\ubar S (\lambda) \rightarrow -\infty$. The result follows from the definition of $\ubar S (\lambda)$.  
\end{proof}

\end{document}
